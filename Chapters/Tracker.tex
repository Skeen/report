\section{Tracker vs. Trackerless}
\label{sec:trackerless}
When running the BitTorrent protocol, we need a bootstrap mechanism, to get in
contact with peers within the network. Several bootstrap mechanisms are exist;
\begin{itemize}
\item Manual peer input
\item Tracker-based peer-lists
\item Trackerless peer-lists
\end{itemize}
The simplest model is that the user provide a number of peers, which are
already within the overlay network. These peers can then by used to further
bootstrap the network (see \citep{bittorrent:bep11}). These pre-known peers
can either be peers connected in the last session (and initially provided by
the software), or found by some means outside the network.

The tracker-based model utilizes a centralized component for acquiring the
initial list of peers to connect to. The flow is simply, a client asks the
tracker for a list of peers given a specific topic (aka. hash), after which the
client is able to contact the peers.

The trackerless model utilizes a \acs{DHT} network for acquiring the initial list of 
peers. The client joins the \acs{DHT} network (by knowing a peer(s) within the network),
and once inside the \acs{DHT} network, it can lookup topics (aka. hashes), and ask
the node responsible for the topic for the peer list.
\newline
- The known peer(s) required to join the \acs{DHT} network, are usually provided by
the \verb|.torrent| file for the download. A reference join node is presented
by BitTorrent, at \url{router.bittorrent.com}, but they recommend running your
own \citep{bittorrent:bep05}.

