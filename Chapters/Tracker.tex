% I have no idea when this fits in
\section{Tracker vs. Trackerless}

When running the BitTorrent protocol, we need a bootstrap mechanism, to get in
contact with peers within the network. Several bootstrap mechanisms are exist;
\begin{itemize}
\item Manual peer input
\item Tracker-based peer-lists
\item Trackerless peer-lists
\end{itemize}
The simplest model is that the user provide a number of peers, which are
already within the overlay network. These peers can then by used to further
bootstrap the network (see \citep{bittorrent:bep11}). These preknown peers
can either be peers connected in the last session (and initially provided by
the software), or found by some means outside the network.

The tracker-based model utilizes a centralized component for acquiring the
initial list of peers to connect to. The flow is simply, a client asks the
tracker for a list of peers given a specific topic (aka. hash), after which the
client is able to contact the peers.

The trackerless model utilizes a DHT network for acquiring the initial list of 
peers. The client joins the DHT network (by knowing a peer(s) within the network),
and once inside the DHT network, it can lookup topics (aka. hashes), and ask
the node responsable for the topic for the peer list.
\newline
- The known peer(s) required to join the DHT network, are usually provided by
the \verb|.torrent| file for the download. A reference join node is presented
by bittorrent, at \url{router.bittorrent.com}, but they recommand running your
own \citep{bittorrent:bep05}.

\subsection{Result}
In our implementation, we've disregarded the manual peer input model, as it is
not user friendly. As such we're left with tracker-based and trackerless, as 
we've limited ourselves to the context of the browser, we're bound to using 
WebRTC, and by extension we need a centralized component to bootstrap these
connections. 

We are however, in theory, free to pick whether we want to use a tracker, or 
utilize a trackerless DHT for bootstrapping peers, but only in theory as our
WebRTC-based BitTorrent implementation (as previously mentioned here) does
indeed not support trackerless, and as the provided WebSocket tracker for 
WebTorrent fulfills both the role of bootstrapping WebRTC connections, and the
role of the BitTorrent tracker.

As previously mentioned, had we been able to utilize trackerless, we've been
able to use the trackerless DHT as our searching DHT, however utilizing a
tracker provides other unique capabilities, namely; centralized control of the
swarm. 
\newline
- A content distributor might like to have this capability, to do data
processing on popular content, or otherwise track the swarm.

While a tracker-based solution utilizes a centralized compontents, it is not
limited to just using one. Our solution utilizes the 4 default trackers
provided by WebTorrent, but it's trivial to run your own tracker and add it to
the resources in the network.
\newline
- A content provider may want to track how often it's content is downloaded,
and could add the content to the network with just one tracker being their own.
