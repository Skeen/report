% Audio Tag capabilities
\section{HTML5 Audio Tag}
Adding support for music file types not already supported by the browser 
would be an major undertaking, 
and would not contribute much towards the goals of this project, 
so we have decided to make use of existing browser capabilities.

% Song Information
\section{Music File-types}
The supported music file types differ greatly from each other internally; 
their audio codings can be different, their meta-data can be different,
and even the way they are rendered and treated by the browsers can differ.
To make matters even worse creators of music files are free to exclude or falsify meta-data, 
and can even leave it out entirely.
Clearly, some kind of common abstraction is needed to encapsulate the uncertain terrain of music files.

To handle false meta-data either an automatic system could be applied, or
community driven verification.

\section{WebRTC limitations}
% WebRTC limitations
WebTorrent (see \ref{sec:WebTorrent}) relies heavily on WebRTC and particularly on WebRTCs dataChannels.
DataChannels are not supported in the Internet Explorer and Safari browsers, 
so these cannot support WebTorrent either.

Creating a data channel is a performance intensive task:
Peers have to contact a signaling service to inform a remote peer of its intention
to make a peer-to-peer connection, both peers have to perform NAT traversal to establish routing information 
and create necessary firewall arrangements, share their routing information,
and finally agree on protocols, encodings and so on.
RTC datachannels have a large overhead when being established,
but can perform low latency communications once established, 
thus we would like to retain datachannels for as long as possible to avoid this overhead.
Many \acs{DHT}, like Kademlia, will establish and discard connections frequently,
and might suffer major overhead when implemented in WebRTC over the datachannel.`

% LocalStorage
\section{Storage}
Ensuring availability of content and robustness in a BitTorrent network requires persistent offline data storage:
without persistent data storage, if all users were to leave the network and return later, all content on the 
network would be lost.
\newline\newline
Until recently the only way to have data persistency through browsers has been cookies, 
which allows saving up to 4KB in each cookie, and can hold at least 50 cookies per domain, 
which allows for a total of 200KB of guaranteed available storage, possibly more depending on the browser.
\newline\newline
Seeing the need for larger amounts of persistent data, browsers now implement Web Storage 
(also known as DOM storage), a new standard by W3C (see \citep{WebStorage})
which consists of sessionStorage and localStorage.

Mozilla also defines a globalStorage for their Firefox browser, 
but this is not supported by any other major browsers, so we will disregard it for this project.

SessionStorage is intended to be wiped whenever the current session ends, 
even if the user agent has not requested it, making it also unsuited for this project.

Data stored using LocalStorage is only wiped when explicitly requested by the user agent or 
by some security policy specified by the user agent beforehand.
\newline\newline
The current Web Storage specification draft is somewhat ambiguous, 
it suggests that browsers implement support for at least 5MB of storage of each kind per domain,
but the actual support varies wildly even between versions of the same browser:

\begin{table}[H]
	\centering
	\begin{tabular}{l | r}
        Browser   & Storage limit \\ \hline
		Chrome 18 & No limit  \\
		Chrome 19 & 1    MB   \\
		Chrome 36 & 5    MB   \\
		Chrome 46 & 4.75 MB   \\
		Firefox   & 5    MB   \\
		IE        & 10   MB   \\
		Safari    & 5    MB   \\
	\end{tabular}
	\caption{Table of size limits of localstorage}
	\label{table:browserls}
\end{table}

Additionally, Firefox allows the user to specify any LocalStorage maximum size, even zero. 
Google have attempted to implement a quota \acs{API} (see \citep{QuotaAPI}), 
which allows websites to ask user agents to increase localstorage limitations,
this has not been adopted by the other browsers, so we had to look for something else.
\newline\newline
In addition to the Web Storage standard, W3C is also working on a standard for IndexedDB in browsers, 
which allows browser-side JavaScript to access a local database on the host machine.

It should be noted that WebTorrent, 
an essential library in our project, 
only officially supports Firefox and Chrome 
both of which do support IndexedDB.

