%% TODO: Billeder

There are two ways to demo our application, namely remotely or locally, we'll
start out explaining the difficult case, namely running the project locally,
and then give an introduction to the easier remotely case.

\subsection{Locally}
We've developed a docker image for our project, as such one can start the 
fake-dht backend server, and the http-server (serving our webinterface), using:
\begin{verbatim}
docker run -it -p 80:1987 -p 3000:3000 skeen/streamy
\end{verbatim}
% TODO: Fix href
After which the webinterface will be hosted on port 80, while the fake-dht,
will be running on port 3000. Assuming the docker instance is running locally,
one should be able to access the webinterface at: \href{http://localhost}.

\subsection{Remotely}
% TODO: Check href
% TODO: Setup DNS for skeen.info
We're running above mentioned docker image on a digital ocean server, as such
one can access the site by going to \href{http://46.101.226.102}.

\subsection{Using the application}
At this point it's assumed that the network is up and running. If one is using 
the local setup, the dht will be empty, and no results will be found by
searching, while two results should pop up using the remote setup, namely;
\begin{itemize}
\item 'The Start': A chiptunes song by Tiasu
\item 'Pixelland': A chiptunes song by Kevin MacLeod
\end{itemize}
Both of these songs are used with permission, and we've setup a node which will
be seeding them continously.

More songs can be added to the system, by dropping them within the drag-and-drop
field on the bottom of the page. When a song is uploaded it is added to the DHT,
saved to local storage, and then seeded by the browser.
% TODO: Fix the issue, that the song isn't added to local before reload
At this point, it should be possible to search for it (using the title), and by
extension downloading them from one browser tab to another.
\newline
- It's preferable to use incognito mode, or different browsers for testing this,
as local storage is shared between tab, and may otherwise interfer.

At this point it's assumed that the dht is populated, and that a seed for the
content is available (note, that the dht does not clear out dead content, or
otherwise decay entries). One should now be able to search for the title of a 
song, clicking it will provide information about the searched content, along 
with the provided information there'll be a button to start downloading the 
content.

Clicking download, will intialize the download and add an entry to the download
tab, which will continously update with information on the download process.

Once the content is fully downloaded it'll move to the local tab, from which it
will be seeded and updated with seeding information. In both the downloading 
and seeding tab, there's a 'add to playlist' button, which appends the song to
the current playlist.

Double-clicking an entry in the playlist starts the playback on the player at
the top of the page. Playing back from the downloads tab setups the torrent in
streaming mode, such that pieces are download in the order they're needed,
rather than in the typical rarest-first order. Note that not all file formats
can be streamed; streamability of different file formats is browser dependent,
and given by the specific implementation of the HTML5 audio tag.

The player contains a varity of buttons, these are annotated on the image below,
but most of these are through to be self explainatory for anyone whom has used
a media player before;
% TODO: Annotated image here.

