% TODO: This goes after an introduction to the architecture of the system.

In our implementation we've utilized serveral third-party libraries, primarily
provided by the node package mananger \verb|npm|. The lists below are excluding
polyfill libraries (i.e. libraries which enable running our code on older
platforms).

For our webinterface, we've utilized;
\begin{itemize}
\item render-media: Which takes \verb|node| audio and video streams, and
        creates / renders to HTML5 audio/video tags.
\item music-metadata: Which takes \verb|node| audio streams, and extracts
        meta-data, such as, but not limited to; artist information, ablum 
        information and song title.
\item express: A HTTP web-server framework, which enabled rapid prototyping and
        development, by easing the definition and implementation of HTTP endpoints.
\item body-parser: A middleware plugin to the express framework, which enables 
        easy access to POST endcoded information.
\item triejs: A pure javascript implementation of a trie datastructure, which 
        enables easy substring lookups.
\item A varity of GUI libraries, in summary:
\begin{itemize}
\item Angular 2: A framework for writing modern web-apps.
\item Font-awesome: A library of glyps and icons.
\item ng2-bootstrap: An Angular 2 wrapper around the css project Bootstrap 3.
\item dragdrop: A smart drag-and-drop element for files, which eases getting
        the HTML5 File object.
\end{itemize}
\end{itemize}
% TODO: Fix these refs
Of these libraries, we've developed patches for 'music-metadata' to add support
for node streams (see \ref{}), and for localforage to add support for node (see
\ref{}).

% TODO: Fix link here
Additionally our webinterface utilizes our own library 'music-streamer-library',
which has been uploaded to the npm repository (see \href{}).

For our music-streamer-library, we've utilized;
\begin{itemize}
\item bittorrent-tracker: A sub-project of WebTorrent, which enables querying 
        the (WebSocket) tracker for seeder and leecher count.
\item localforage: Which is a unified adapter ontop of the varying HTML5
        storage backends, such as: LocalStorage, IndexedDB and WebSQL.
\item magnet-uri: Which parses magnet URIs and output JSON objects containing
        the information from within the magnet URI.
\item request: A small library that simplifies sending XHR (HTTP requests).
\item readable-blob-stream: Which generates node streams from HTML5 Blob objects.
\item WebTorrent: A full BitTorrent client, implemented using WebRTC as the 
        transport layer. This is used for the actual data transfer.
\item Additionally; Our library utilizes some of the same libraries as our
        webinterface, namely: musicmetadata and render-media.
\end{itemize}

% TODO: Fix this ref
Our music-streamer-library, contains an interface for a distributed hash table,
for which we have worked on creating multiple implementations. Our intial one,
was a simple centralized lookup table server, utilized to get the rest of the 
project going. For more information about the DHT implementations we've 
investigated, please refer to \ref{}.

% TODO: Fix this ref
% TODO: Better word than richer
The DHT implementation utilized in the demo (see \ref{}), is a richer version
of our initial lookup table server implementation, we've named this; 'fake-dht'.

For our fake-dht centralized lookup / searching server, we've utilized;
\begin{itemize}
\item express: A HTTP web-server framework, which enabled rapid prototyping and
        development, by easing the definition and implementation of HTTP endpoints.
\item body-parser: A middleware plugin to the express framework, which enables 
        easy access to POST endcoded information.
\item triejs: A pure javascript implementation of a trie datastructure, which 
        enables easy substring lookups.
\end{itemize}
