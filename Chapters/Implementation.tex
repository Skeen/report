\section{Overall Architecture}
% Code Architecture
Due to the size of the projects code-base and large number of libraries with different \acs{API}s, 
it was necessary to create different object classes with separated concerns,
and a consistent way to use them.

The code base is separated into a backend with code parts that are not visible to the user,
and a front end with each class having an associated \acs{GUI} element.
We also needed main classes to coordinate these responsibilities, handle overall flow,
and serve as a container for anything we did not have time to place in a separate class:
these classes are lib.ts in the backend,
and app.component.ts in the front-end to hold \acs{GUI} elements.

Many of the libraries used have slow, non-blocking IO operations
and make use of callbacks to execute code after the IO operations complete.

Most \acs{GUI} actions result in an event rather than an immediate code execution, 
these events are then caught by app.component
which performs the necessary functionality, through the other classes when relevant.

\begin{table}[h]
	\centering
	\begin{tabular}{l | l}
		changing\_song   & User selected a different song to be played \\
		nextSong         & User clicked the next song button \\
		prevSong         & User clicked the previous song button \\
		song\_ended      & Song has ended and in line next should be played \\
		downloaded       & Song has finished downloading \\
		ready-for-seed   & Song is ready to be seeded \\
		add-song         & Song added to playlist (from downloads) \\
		add-song         & Song added to playlist (from localcontent) \\
		drop-down-select & User clicked a search result \\
	\end{tabular}
	\caption{Events used in Streamy}
	\label{table:events}
\end{table}

When the system is started by the browser,
Angular loads the GUI elements and then calls ngAfterViewInit in app.component.
ngAfterViewInit performs event bindings, and starts localcontent seeding.
Localcontent retrieves songs stored on the disk, and starts seeding them afterwards using a callback.
All other system functionality is triggered via these events, 
which are driven mostly by user actions and the music player.
\newline

% \acs{GUI} (Angular)
The \acs{GUI} needed data-bindings to show torrent information, allow changing songs
and implement a music player \acs{GUI}. 
Angular 2 lets us create \acs{GUI} elements as new \acs{HTML} tags, 
and bootstrap provides generic \acs{GUI} elements and stylesheets for angular, 
they were the easiest way to fill our needs for the \acs{GUI},
so that is what we have used.

Our \acs{GUI} elements consist of a music player interface, a play-list, downloading information 
and show information about local content and seeding.
\newline

\section{How we use Browserify}
% Browserify and why we use it
The project contains a large amount of NodeJS module libraries, 
many of which are designed for NodeJS and not for use in browsers.
We needed some way to use these libraries, and to insure that they are properly loaded.
In NodeJS, modules are loaded and made available by calling the require method 
and assigning the result to a field variable, which can then be used to access the libraries functions.
This capability is not supported natively in browsers, so a tool was needed to provide it.
The Browserify tool takes JavaScript code written for NodeJS, 
transforms it into regular JavaScript code, 
handles all the requires, 
and emits the new JavaScript code as a single easily distributed file called Bundle.js.
Browserify allows our projects end result to consist only of a \acs{HTML} page and one JavaScript file, 
so this seems like an ideal setup, and is what we make use of.
\newline

\section{Song handling}
% How we handle music and music metadata
The program encapsulates the highly inconsistent music files using a Song object,
which contains optional fields corresponding to commonly used meta-data, 
such as title, name of artist, genre and album. 
The Song object also contains information relevant for the BitTorrent system; 
it contains the magnetURI it was retrieved from or is being seeded to, 
it holds the binary file once it is fully available, 
or a reference to the music stream when it is still being downloaded.
This Song object presents all other sections of the project with a uniform way to access meta-data and audio, 
while retaining relational information about which albums or artists the song belongs to. In addition to this, 
we have also created objects to hold information about an album or an artist, 
so we can more easily search for related works in the \acs{DHT}.
\newline

\section{Torrenting}
% WebTorrent
WebTorrent provides seed and download methods using MagnetURI.
Individual downloads and seeds are controlled through a Torrent object provided by WebTorrent,
which contains all the relevant information about progress, 
speed, amount uploaded, number of peers, contained files and so on.
These torrent objects are provided by callbacks at events
and after each called WebTorrent function completes.
The encapsulating torrent.ts class gives access to seed and download methods of WebTorrent, 
but only for our Song Objects, which emphasis the projects intent of being a music streaming service 
rather than generic file sharing application, 
it also provides the WebTorrent torrent object through callbacks to other sections of our code.
\newline

\section{LocalContent \& Storage}
% Storage
The object-blob separation behavior and use of localforage,
needed to be consistent throughout the project, 
so we created a Storage class which uses localforage and handles the data separation implicitly when its storage methods are called.
We have also disallowed the direct use of localforage anywhere else:
all other sections of the project should save and get data through using the Storage class to ensure consistent behavior. 
This was done by removing any inclusion of the localforage library outside of the Storage class file.

As local content should also be visible in the \acs{GUI}, 
be seeded and be capable of being added to the playing,
we created another class called LocalContent,
which retrieves all songs from the disk at start-up,
begins to seed them, making them available for other users in the network to download,
and presents a \acs{GUI} element to the user.
