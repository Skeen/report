Due to the size of the projects codebase and large number of libraries, it was necessary to create a maintainable architecture.

% Browserify and why we use it
The project contains a large amount of NodeJS module libraries, 
many of which are designed for NodeJS and not for use in browsers.
We needed some way to use these libraries, and to insure that they are properly loaded.
In NodeJS, modules are loaded and made available by calling the require method 
and assigning the result to a field variable, which can then be used to access the libraries functions.
This capability is not supported natively in browsers, so a tool was needed to provide it.
The Browserify compile tool takes NodeJS code and emits a bundle.js, 
which includes all the input code and needed libraries. 


% How we handle music and music metadata

% WebTorrent

% Code Architecture

% Storage

The object-blob seperation behavior and use of localforage,
needed to be constistent throughout the project, 
so we created a Storage class which uses localforage and handles the data seperation implicitly, 
and disallowed the direct use of localforage anywhere else: all other sections of the project should save and get data through using the Storage class.

As local content should also be playable, 
seeded and be capable of being added to the playlist, 
we created another class called localcontent,
which retrieves all songs from the disk at startup,
begins to seed them, making them available for other users in the network to dowbload,
and presents a GUI element to the user.

% GUI (Angular)

