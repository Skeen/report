With this project work, we wanted to test whether it was feasible to implement
completely distributed content distribution network, i.e. that we can combine
the content indexing and control of centralized distribution networks, with a
distributed data sharing network, such as BitTorrent, and thus produce a system
with the qualities of both the centralized and the distributed systems.
\newline\newline
As we have not been able to construct a completely distributed system, given
the constraints of our operating environment (the context of a web-browser), we
do not believe we can confirm our hypothesis.

We do however provide our system as evidence for our hypothesis, since we have
succesfully created such a system, by utilizing a centralized server component,
with an interface which is provably implementable using distributed techniques.
\newline\newline
Due to time constaints, we never managed to implement replication seeding, and
as such, we have not been able to run experiments on replications factors, such
as availability, robustness and fairness. Additionally we've been unable to
test out different eviction strategies for storage removing songs from
localstorage.
\newline\newline
We did however conducted experiments on the scalability and performance of the 
system in terms of memory and CPU usage.

We've found that our memory usage scales linearily in the number of songs added
to the client, as such we conclude that our system is scalable with regards to
local memory usage. In order to verify the linearity in real world use cases,
we have tested and confirmed that songs of different sizes adhere to the same
overhead factor of about $\sim6.5$.
\newline\newline
In conclusion; our system is completely usable and performs well, and while it 
does lack a few features for completeness, we believe that simply replacing our
our faked \acs{PHT} implemenation with a fully distributed one, would suffice 
for proving our hypothesis; this is however, due to time constraints, left as 
future work.
