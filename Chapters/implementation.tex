Due to the size of the projects codebase and large number of libraries, it was necessary to create a maintainable architecture.


% Section for Torrent

% Section for Storage
To have persistency in user sessions, we needed a way of storing data to the disk. There are several standard ways to store data in web browsers, but all of these have their own drawbacks.

Traditional browser cookies can store up to 4KB of data, 
not enough to save even one song. 
To store a song in cookies, 
we would need to create thousands of cookies, 
and store parts of a song in each one, 
this seemed like a difficult and inelegant solution.

Seeing the need for larger storage spaces, 
industry concerns have added LocalStorage to the HTML 5 standard,
which allows websites to store arbitrarily large amounts of data,
but this again suffers significant drawbacks: 
implementation varies greatly between browsers, 
and can in some cases be wiped after reaching only 5 megabytes, 
hardly enough for our purposes.

Some browsers have also added support for database storage, via IndexedDB and mysql, but this is not supported in every browser.

To overcome these challenges, we use a library called localForage, 
which checks what the host browser supports,
and makes use of the best available storage method.
We found through experiments,
that localForage has issues saving javascript objects to disk when these objects contain large binary blobs, 
these could consume as much as 4 gigabytes of space.
Seperating the generated metadata object and song binary before placing the in storage, 
seemed to solve this issue.

The object-blob seperation behavior and use of localforage,
needed to be constistent throughout the project, 
so we created a Storage class which uses localforage and handles the data seperation implicitly, 
and disallowed the direct use of localforage anywhere else: all other sections of the project should save and get data through using the Storage class.

% Section for Song
