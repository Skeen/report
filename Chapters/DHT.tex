% TODO: Describe our DHT interface

% This goes after the libraries section
As previously mentioned, we've implemented an interface for our distributed
hash table implementations. The current implementation fulfills the interface
by utilizing a centralized component, this is sub-optimal in what is declared
as a distributed system, below is an argumentation as of why this is the case.

During our project development, we've investigated a few different WebRTC-based
distributed hash table implementations, namely;
\begin{itemize}
\item WebRTC-explorer
\item WebRTC-chord
\item Kad-WebRTC
\end{itemize}

\subsection{WebRTC-explorer}
Our experiments with WebRTC-explorer was mostly demotivation, the project does
not implement the usual DHT interface, but rather a network overlay structure,
upon which the (single) developer suggests that one can implement a DHT network.

The project is currently in development, by a masters student and the quality,
and stability of the project seems to propagate this.

\subsubsection{WebRTC-chord}
This project is created as a precursor for WebRTC-explorer, and was through was
a direct implementation of the chord network, using WebRTC, however during 
development the above mentioned masters student found that implementing chord,
requires a lot of (WebRTC) connections, i.e. for successor lists and
finger-tables, and as such he hit the connection limit imposed by the browser.

This was in essence the issue, which caused him to develop a new network
overlay structure; the one utilized by WebRTC-explorer. As a result, this 
project has remained unmaintained for over two years.

\subsection{Kad-WebRTC}
% TODO: Fix these refs
Where's the previous two projects are standalone, this is no more than a 
transportations layer plugin for the Kad project (see \ref{}). As such it 
has the same interface as the Kad project for desktops, which is indeed the 
expected DHT interface. The WebRTC transport layer utilizes a centralized 
component to setup the WebRTC connections (see \ref{}), during our experiments
this was somewhat easy to set up and get running, but no interactive example
was present in the project.

We've developed an interactive example for Kad-WebRTC as a part of our
experimententation, and pushed this upstream, such that it is now included in 
the official repository, see \ref{} for details.

% TODO: Document UDP for Kademlia
While this project did indeed fulfill our expectations, and implements the DHT
interface, it is not without it's problems. In truth, we did not expect the 
project to actually run, as the Chord WebRTC implemtation mentioned above hits
the limit for WebRTC connections, and since Kademlia generally utilizes a huge
amount of of connections (Kademlia is usually implemented using UDP).

When we investigated this further, we found that this WebRTC transport layer,
is simply used as a plug-in replacement for UDP, and as such is connectionless.
Whenever a connection is required, a new one is spawned, this obviously
overcomes the browser limitations on the number of WebRTC connections, but 
instead it establishes new connections constantly.

As previously described WebRTC connections are setup using a centralized server,
and each connection requires in the order of 10 messages total to set up.
Additionally we've found that resyncronization of key-value pairs takes a lot
of time, as Kad is implemented upon the concept of eventual consistency. This
conflicts with our needs, and as such we deemed this project unfeasible for
our choice of DHT.

\subsection{Alternatives}
At this point we discussed alternatives for how to implement our lookup /
searching;
\begin{itemize}
\item Centralized searching
\item Searching in unstructured networks
\item Look for more DHTs?
\end{itemize}

Utilizing a centralized searching component would eliminiate the purpose of our
project, as our project would only prove the feasibility of using BitTorrent as
a data transfer technology under a content distribution network, which has been
proven rigiously before, by the common adaptation of BitTorrent.

% TODO: Cite gnutella
% TODO: inscalable??
We looked into searching in unstructured peer-2-peer networks, but most
literature seems to point towards using some sort of structure for lookup and 
searching. If we were to pick this model we'd have implemented a unstructued 
network utilizing $k$-random-walkers for searching, as broadcasting would be 
inscalable \citep{}

While searching for alternative DHTs, we came across the BitTorrent BEP44 (see
\ref{}). In summary, this allows using the trackerless BitTorrent DHT for 
storing arbitrary data. While WebTorrent does implement this protocol on 
desktop (i.e. in node using UDP/TCP), it does not implement the protocol using
WebRTC, and as such cannot be used within the context of the web browser.

\subsection{Result}
As there does not seem to be a WebRTC DHT implementation, which fulfills our 
requirements, and since we're unable to implement one ourselves within the 
timeframe of the project, we've decided to continue utilizing our dht-fake,
under a few specific limitations.

The centralized component cannot have a richer interface than a normal
distributed hash table, and everything the fake does, should be implementable
using a WebRTC-based DHT.

% TODO: PHT paper
This meant that throughout the most of our project, we did exact-match
searching, that is until we found \citep{}, which described implementing
substring searching directly on top an arbitary distributed hash table.

% TODO: Verify claim
The publisher of this paper does not release their implementation, and as such
we've not been able to verify the claims they make, but if the claims hold, it
seems fit for our use-case.
