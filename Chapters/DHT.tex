% This goes after the libraries section
\section{\acs{DHT}}
\label{sec:dht}

As previously mentioned, our \verb|music-streamer-library| features an
interface for distributed hash tables. This interface is simple, and features
just two methods; namely:
\begin{itemize}
    \item get: Which given a hash looks up the responsible node, and gets the 
        stored information.
    \item set: Which given a hash and a value, looks up the the responsible
        node, and asks it to store the value.
\end{itemize}
This interface was by design created to resembles a non-distrbuted hash table,
as the distributivity of the underlying implementation is to be considered an 
implementation detail.

Evindence for the feasability of just leaving this as an implementation detail,
is provided as the authors of this report has previously implemented this exact
interface, on top of a Chord overlay network (see \citep{Skeen:Chord}).
Additionally, the creators of Chord suggest exactly a \acs{DHT} as one of the example
applications of their network (see \citep{Brunskill:Chord}).

This even through Chord does only support a single operation in it's public 
interface, namely: \verb|lookup|, which lookup the node responsible for a
specific hash / ID (for details about Chord, see \citep{Stoica:Chord}).
\newline\newline
Our current implementation fulfills the above interface by utilizing a
centralized server component, this is clearly sub-optimal in what is declared
as a distributed system; below is an argumentation as of why this is the case.

During our project development, we've investigated a few different WebRTC-based
distributed hash table implementations, namely;
\begin{itemize}
    \item WebRTC-explorer (see \citep{diasdavid:webrtc-explorer})
    \item WebRTC-chord (see \citep{diasdavid:webrtc-chord})
    \item Kad-WebRTC (see \citep{kadtools:kad-webrtc})
\end{itemize}
As we found none of these systems, fulfilled our expectations, we ended up 
considering other alternatives; A discussion of these additional alternatives,
is emitted directly after a discussion of the above systems.

\subsection{WebRTC-explorer}
Our experiments with WebRTC-explorer were generally demotivating, the project
does not implement the usual \acs{DHT} interface, but rather a Chord-like
\verb|lookup| network overlay structure interface, as with Chord it's suggested
that one can implement a \acs{DHT} upon this overlay network interface.
\newline
The project is currently in development, by a single masters student; It is our
verdict that the quality, and stability of the project seems to propagate this.

The instability of the project, combined with time-constaints of the project
had us drop this as a possible \acs{DHT} implementation.

\subsubsection{WebRTC-chord}
This project is created as a precursor for WebRTC-explorer, and was through as
a direct implementation of the chord network, using WebRTC. However during 
development the above mentioned masters student found that implementing chord,
requires a lot of (WebRTC) connections, i.e. for successor lists and
finger-tables, and as such he hit the connection limit imposed by the browser;

\begin{displayquote}
    DISCLAIMER: Implementing a full Chord algorithm proved to be an unviable option
    (due to the high number of data-channels that had to be open inside a browser),
    in order to overcome this, I've come up with webrtc-explorer, and adaptable
    Chord like implementation. I'll no longer support this module \\ \medskip
    --- David Dias
\end{displayquote}
Due to this, and the fact that the project has now remained unmaintained for
over two years, we decided against using this as our \acs{DHT} implementation.
\newline\newline
The same developer has previously created a project called \verb|webrtc-ring|
(see \citep{diasdavid:webrtc-ring}), it's deprecated like WebRTC-chord in favor
of WebRTC-explorer, we have therefore not been investigating this project
further.

\subsection{Kad-WebRTC}
Where's the previous two projects are standalone, this is no more than a 
transportations layer plugin for the Kad project (see \citep{kadtools:kad}).
As such it has the same interface as the Kad project for desktops, which is
indeed the  expected \acs{DHT} interface. The WebRTC transport layer utilizes a
centralized component to setup the WebRTC connections (see
\ref{webrtc-connection-server}), during our experiments this was somewhat easy
to set up and get running, but no interactive example was present in the
project.

We've developed an interactive example for Kad-WebRTC as a part of our
experimententation, and pushed this upstream, such that it is now included in 
the official repository (see \ref{subsec:appendix-kad-webrtc} for details).

While this project did indeed fulfill our expectations, and implements our \acs{DHT}
interface, it is not without it's problems. In truth, we did not expect the 
project to actually run, as the Chord WebRTC implemtation mentioned above hits
the limit for WebRTC connections, and since Kademlia generally utilizes a huge
amount of of connections (The official paper on Kademlia, suggests using \acs{UDP}
(see \citep{Maymounkov:Kademlia})).

This sparked our interest, as to how this project could avoid the limit, and as
we investigated this further, we found that this WebRTC transport layer, is
simply used as a plug-in replacement for \acs{UDP}, and as such is connectionless.
Whenever a connection is required, a new one is spawned, this obviously
overcomes the browser limitations on the number of WebRTC connections, but 
instead it establishes new connections constantly.

As previously described WebRTC connections are setup using a centralized server,
and each connection requires in the order of 10 messages total to set up.
Additionally we've found that resyncronization of key-value pairs takes a lot
of time, as Kad is implemented upon the concept of eventual consistency. This
conflicts with our needs, and as such we deemed this project unfeasible for
our choice of \acs{DHT}.

\subsection{\acs{DHT} alternatives}
At this point; we were close to given up our search for a WebRTC \acs{DHT}
implementation, and thus we discussed several alternatives for how to
implement our lookup / searching;
\begin{itemize}
\item Centralized searching
\item Searching in unstructured networks
\item Continue searching for \acs{DHT}s
\end{itemize}

\subsubsection{Centralized searching}
Utilizing a centralized searching component would eliminiate the purpose of our
project, as our project would only prove the feasibility of using BitTorrent as
a data transfer technology under a content distribution network, which has been
proven rigiously before, by the common adaptation of BitTorrent.

As such we abandoned this idea again quite quickly.

\subsubsection{Searching in unstructured networks}
We looked into searching in unstructured peer-2-peer networks, there are
several options for implementing this, amongst the first ones is flooding
popularized by the Gnutella network (see \ref{sec:gnutella}).

Flooding does ensure that if a piece of informating is contained within the 
network it will be found, however it does not scale well and searching 
generates a lot of traffic (see \citep{Lv:Searching}).
\newline
There are however ways to improve upon flooding, such as described by Gia 
(see \citep{Chawathe:Gia}), where the developers claim to have improved upon
the naive flooding of Gnutella, by three to five orders of magnitude.
They achieve this improvement by the synergy of different modifications,
including flow control, topology adaptation, replication and attention to node
heterogeneity.
\newline\newline
An alternative to flooding, is the $k$-random-walkers algorithm, in which a 
node initiates a search by querying $k$ random neighbours for it, these
neighbours then forward the request to a single of their neighbours, until a 
match is found or a TTL (time to live) variable dries out.

The random-walker approach reduces network usage by two orders of magnitude,
compared to the naive flooding (see \citep{Lv:Searching}), and therefore it is
a lot more scalable, and could be a feasible solution; indeed if it was
combined with replication.
\newline\newline
Due to time constaints we decided against looking further into using an
unstructured peer-2-peer network for searching. In retrospective, we should
have been more open-minded towards different network structures before deciding
upon using a distributed hash table interface.

The Gia paper (see \citep{Chawathe:Gia}) contains an excellent discussion about
unstructured peer-2-peer network vs. structured ones in regards to searching,
the main points being; \acs{DHT}s generally have a harder time in regards to dealing
with churn and handling keyword seaches (\acs{DHT}s generally implement exact match
lookup only).

\subsubsection{Continue searching for \acs{DHT}s}
As we rejected the idea of using an unstructured peer-2-peer network for
searching, we decided to continue our search for WebRTC enabled \acs{DHT}s. Which
brought us to BitTorrent \acs{BEP}44 (see \citep{bittorrent:bep44}). In summary, this
protocol extension enables using the trackerless BitTorrent \acs{DHT} for storing
arbitrary data. While WebTorrent does implement this protocol on desktop (i.e.
in node using \acs{UDP}/\acs{TCP}), it does not implement the protocol using WebRTC, and as
such cannot be used within the context of the web browser, and by extension
within our application.
\newline\newline
Within the given time constaints we have not been able to find a WebRTC enabled
\acs{DHT} implementation, which fulfills our requirements and expectations, and since
this very same time-frame limits us from implementing one ourselves, we've been
forced to stick with our fake implementation.

\subsection{\acs{DHT}-/\acs{PHT}-fake}
As mentioned in the previos section, we've been forced to stick with our faked
implementation, that is we've been forced to use a centralized server component,
to justify this in regards to our hypothesis; we've limited ourselves by the 
following rules;
\begin{itemize}
\item The centralized server component cannot have a richer interface than a 
    normal distributed hash table.
\item Everything our fake does, should be implmentable using a WebRTC-based \acs{DHT}.
\end{itemize}
As such, throughout most of our project, we've been relying solely on
exact-match lookup, i.e. as described as a limitation of \acs{DHT}s by the Gia paper
(see \citep{Chawathe:Gia}). We did however stuble upon a paper implementing 
sub-string searching on top of \acs{DHT}s, by implementing a hash-based trie (\acs{PHT})
(see \citep{Ramabhadran:PHT}).

During our continued investigation of the literature within the domain, we 
found another paper, which claims to have improved upon the \acs{PHT} paper, by 
minimizing the maintainance cost of the trie structure by up to 75\%. Their
paper introduces a new data structure called a \ac{LHT},
which alike the \acs{PHT} is built ontop of \acs{DHT}s (see \citep{Tang:LHT}).

The publishers of these papers does not release their implementation, and as
such we've not been able to verify their claims, but assuming they do hold the
\acs{PHT} / \acs{LHT} seems fit for our use-case. As such we've expanded our \verb|dht-fake|,
to become a \verb|pht-fake|, i.e. to support substring searching.
