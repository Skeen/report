% This goes after the libraries section
\section{DHT}
\label{sec:dht}

As previously mentioned, our \verb|music-streamer-library| features an
interface for distributed hash tables. This interface is simple, and features
just two methods; namely:
\begin{itemize}
    \item get: Which given a hash looks up the responsible node, and gets the 
        stored information.
    \item set: Which given a hash and a value, looks up the the responsible
        node, and asks it to store the value.
\end{itemize}
This interface was by design created to resembles a non-distrbuted hash table,
as the distributivity of the underlying implementation is to be considered an 
implementation detail.

Evindence for the feasability of just leaving this as an implementation detail,
is provided as the authors of this report has previously implemented this exact
interface, on top of a Chord overlay network (see \citep{Skeen:Chord}).
Additionally, the creators of Chord suggest exactly a DHT as one of the example
applications of their network (see \citep{Brunskill:Chord}).

This even through Chord does only support a single operation in it's public 
interface, namely: \verb|lookup|, which lookup the node responsible for a
specific hash / ID (for details about Chord, see \citep{Stoica:Chord}).
\newline\newline
Our current implementation fulfills the above interface by utilizing a
centralized server component, this is clearly sub-optimal in what is declared
as a distributed system; below is an argumentation as of why this is the case.

During our project development, we've investigated a few different WebRTC-based
distributed hash table implementations, namely;
\begin{itemize}
    \item WebRTC-explorer (see \citep{diasdavid:webrtc-explorer})
    \item WebRTC-chord (see \citep{diasdavid:webrtc-chord})
    \item Kad-WebRTC (see \citep{kadtools:kad-webrtc})
\end{itemize}
As we found none of these systems, fulfilled our expectations, we ended up 
considering other alternatives; A discussion of these additional alternatives,
is emitted directly after a discussion of the above systems.

\subsection{WebRTC-explorer}
Our experiments with WebRTC-explorer were generally demotivating, the project
does not implement the usual DHT interface, but rather a Chord-like
\verb|lookup| network overlay structure interface, as with Chord it's suggested
that one can implement a DHT upon this overlay network interface.
\newline
The project is currently in development, by a single masters student; It is our
verdict that the quality, and stability of the project seems to propagate this.

The instability of the project, combined with time-constaints of the project
had us drop this as a possible DHT implementation.

\subsubsection{WebRTC-chord}
This project is created as a precursor for WebRTC-explorer, and was through as
a direct implementation of the chord network, using WebRTC. However during 
development the above mentioned masters student found that implementing chord,
requires a lot of (WebRTC) connections, i.e. for successor lists and
finger-tables, and as such he hit the connection limit imposed by the browser;

\begin{displayquote}
    DISCLAIMER: Implementing a full Chord algorithm proved to be an unviable option
    (due to the high number of data-channels that had to be open inside a browser),
    in order to overcome this, I've come up with webrtc-explorer, and adaptable
    Chord like implementation. I'll no longer support this module \\ \medskip
    --- David Dias
\end{displayquote}
Due to this, and the fact that the project has now remained unmaintained for
over two years, we decided against using this as our DHT implementation.
\newline\newline
The same developer has previously created a project called \verb|webrtc-ring|
(see \citep{diasdavid:webrtc-ring}), it's deprecated like WebRTC-chord in favor
of WebRTC-explorer, we have therefore not been investigating this project
further.

\subsection{Kad-WebRTC}
Where's the previous two projects are standalone, this is no more than a 
transportations layer plugin for the Kad project (see \citep{kadtools:kad}).
As such it has the same interface as the Kad project for desktops, which is
indeed the  expected DHT interface. The WebRTC transport layer utilizes a
centralized component to setup the WebRTC connections (see
\ref{webrtc-connection-server}), during our experiments this was somewhat easy
to set up and get running, but no interactive example was present in the
project.

We've developed an interactive example for Kad-WebRTC as a part of our
experimententation, and pushed this upstream, such that it is now included in 
the official repository (see \ref{subsec:appendix-kad-webrtc} for details).

While this project did indeed fulfill our expectations, and implements our DHT
interface, it is not without it's problems. In truth, we did not expect the 
project to actually run, as the Chord WebRTC implemtation mentioned above hits
the limit for WebRTC connections, and since Kademlia generally utilizes a huge
amount of of connections (The official paper on Kademlia, suggests using UDP
(see \citep{Maymounkov:Kademlia})).

This sparked our interest, as to how this project could avoid the limit, and as
we investigated this further, we found that this WebRTC transport layer, is
simply used as a plug-in replacement for UDP, and as such is connectionless.
Whenever a connection is required, a new one is spawned, this obviously
overcomes the browser limitations on the number of WebRTC connections, but 
instead it establishes new connections constantly.

As previously described WebRTC connections are setup using a centralized server,
and each connection requires in the order of 10 messages total to set up.
Additionally we've found that resyncronization of key-value pairs takes a lot
of time, as Kad is implemented upon the concept of eventual consistency. This
conflicts with our needs, and as such we deemed this project unfeasible for
our choice of DHT.

\subsection{Alternatives}
At this point; we were close to given up our search for a WebRTC DHT
implementation, and thus we discussed several alternatives for how to
implement our lookup / searching;
\begin{itemize}
\item Centralized searching
\item Searching in unstructured networks
\item Continue searching for DHTs, or write our own?
\end{itemize}

\subsubsection{Centralized searching}
Utilizing a centralized searching component would eliminiate the purpose of our
project, as our project would only prove the feasibility of using BitTorrent as
a data transfer technology under a content distribution network, which has been
proven rigiously before, by the common adaptation of BitTorrent.

As such we abandoned this idea again quite quickly.

\subsubsection{Searching in unstructured networks}

% TODO: Cite gnutella
% TODO: inscalable??
We looked into searching in unstructured peer-2-peer networks, but most
literature seems to point towards using some sort of structure for lookup and 
searching. If we were to pick this model we'd have implemented a unstructued 
network utilizing $k$-random-walkers for searching, as broadcasting would be 
inscalable \citep{}

While searching for alternative DHTs, we came across the BitTorrent BEP44 (see
\ref{}). In summary, this allows using the trackerless BitTorrent DHT for 
storing arbitrary data. While WebTorrent does implement this protocol on 
desktop (i.e. in node using UDP/TCP), it does not implement the protocol using
WebRTC, and as such cannot be used within the context of the web browser.

\subsection{Result}
As there does not seem to be a WebRTC DHT implementation, which fulfills our 
requirements, and since we're unable to implement one ourselves within the 
timeframe of the project, we've decided to continue utilizing our dht-fake,
under a few specific limitations.

The centralized component cannot have a richer interface than a normal
distributed hash table, and everything the fake does, should be implementable
using a WebRTC-based DHT.

% TODO: PHT paper
This meant that throughout the most of our project, we did exact-match
searching, that is until we found \citep{}, which described implementing
substring searching directly on top an arbitary distributed hash table.

% TODO: Verify claim
The publisher of this paper does not release their implementation, and as such
we've not been able to verify the claims they make, but if the claims hold, it
seems fit for our use-case.
