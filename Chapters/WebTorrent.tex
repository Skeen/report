\section{WebTorrent}
\label{sec:WebTorrent}
WebTorrent is one of the two primary technology used within our project (along 
side with the \acs{DHT}), and as such it deserves it's own treatment, within the 
report.
\newline\newline
WebTorrent is as previously mentioned a full fledged BitTorrent implementation, 
utilizing WebRTC as the underlying transportation layer technology. While it 
does in fact implement both the usual and the WebRTC transportation layers,
we'll only discuss the WebRTC part, as that's what's interesting for our
use-case.
\newline
WebTorrent provides a somewhat simple interface, with the two main methods being;
\begin{itemize}
    \item add; Which takes a \verb|torrentID| (i.e. a torrent file or magnetURI),
        and starts downloading it's content; firing a callback once meta-data has
        been acquired.
    \item seed; Which takes a \verb|File|, and starts seeding it; firing a
        callback once it's seeding.
\end{itemize}
WebTorrent supports steaming of torrent, and this is automatically enabled, if 
one accesses the file of a torrent via. read stream (see \citep{WebTorrent:api}).
