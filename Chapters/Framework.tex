% TODO: This goes after an introduction to the architecture of the system.
\section{Libraries}

In our implementation we've utilized serveral third-party libraries, primarily
provided by the node package mananger \verb|npm|. The lists below are excluding
polyfill libraries (i.e. libraries which enable running our code on older
platforms).

For our webinterface, we've utilized;
\begin{itemize}
\item render-media: Which takes \verb|node| audio and video streams, and
        creates / renders to HTML5 audio/video tags.
\item music-metadata: Which takes \verb|node| audio streams, and extracts
        meta-data, such as, but not limited to; artist information, ablum 
        information and song title.
\item express: A HTTP web-server framework, which enabled rapid prototyping and
        development, by easing the definition and implementation of HTTP endpoints.
\item body-parser: A middleware plugin to the express framework, which enables 
        easy access to POST endcoded information.
\item triejs: A pure javascript implementation of a trie datastructure, which 
        enables easy substring lookups.
\item A varity of GUI libraries, in summary:
\begin{itemize}
\item Angular 2: A framework for writing modern web-apps.
\item Font-awesome: A library of glyps and icons.
\item ng2-bootstrap: An Angular 2 wrapper around the css project Bootstrap 3.
\item dragdrop: A smart drag-and-drop element for files, which eases getting
        the HTML5 File object.
\end{itemize}
\end{itemize}
% TODO: Fix these refs
Of these libraries, we've developed patches for 'music-metadata' to add support
for node streams (see \ref{}), and for localforage to add support for node (see
\ref{}).

% TODO: Fix link here
Additionally our webinterface utilizes our own library 'music-streamer-library',
which has been uploaded to the npm repository (see \href{}).

For our music-streamer-library, we've utilized;
\begin{itemize}
\item bittorrent-tracker: A sub-project of WebTorrent, which enables querying 
        the (WebSocket) tracker for seeder and leecher count.
\item localforage: Which is a unified adapter ontop of the varying HTML5
        storage backends, such as: LocalStorage, IndexedDB and WebSQL.
\item magnet-uri: Which parses magnet URIs and output JSON objects containing
        the information from within the magnet URI.
\item request: A small library that simplifies sending XHR (HTTP requests).
\item readable-blob-stream: Which generates node streams from HTML5 Blob objects.
\item WebTorrent: A full BitTorrent client, implemented using WebRTC as the 
        transport layer. This is used for the actual data transfer.
\item Additionally; Our library utilizes some of the same libraries as our
        webinterface, namely: musicmetadata and render-media.
\end{itemize}

% TODO: Fix this ref
Our music-streamer-library, contains an interface for a distributed hash table,
for which we have worked on creating multiple implementations. Our intial one,
was a simple centralized lookup table server, utilized to get the rest of the 
project going. For more information about the DHT implementations we've 
investigated, please refer to \ref{}.

% TODO: Fix this ref
% TODO: Better word than richer
The DHT implementation utilized in the demo (see \ref{}), is a richer version
of our initial lookup table server implementation, we've named this; 'fake-dht'.

For our fake-dht centralized lookup / searching server, we've utilized;
\begin{itemize}
\item express: A HTTP web-server framework, which enabled rapid prototyping and
        development, by easing the definition and implementation of HTTP endpoints.
\item body-parser: A middleware plugin to the express framework, which enables 
        easy access to POST endcoded information.
\item triejs: A pure javascript implementation of a trie datastructure, which 
        enables easy substring lookups.
\end{itemize}


% This goes after the libraries section
\section{DHT}
\label{sec:dht}

As previously mentioned, our \verb|music-streamer-library| features an
interface for distributed hash tables. This interface is simple, and features
just two methods; namely:
\begin{itemize}
    \item get: Which given a hash looks up the responsible node, and gets the 
        stored information.
    \item set: Which given a hash and a value, looks up the the responsible
        node, and asks it to store the value.
\end{itemize}
\begin{lstlisting}[caption={TypeScript DHT Hash-Table interface}]
export interface HashTable
{
    // Put a keyset into the HashTable, callback when done
    put(key:string, value:string, callback?:DHTCallback) : void;
    // Get a value from the HashTable, callback when done
    get(key:string, callback?:DHTCallback) : void;
}
\end{lstlisting}
This interface was by design created to resembles a non-distributed hash table,
as the distributivity of the underlying implementation is to be considered an 
implementation detail.

Evidence for the feasibility of just leaving this as an implementation detail,
is provided as the authors of this report has previously implemented this exact
interface, on top of a Chord overlay network (see \citep{Skeen:Chord}).
Additionally, the creators of Chord suggest exactly a \acs{DHT} as one of the example
applications of their network (see \citep{Brunskill:Chord}).

This even through Chord does only support a single operation in it's public 
interface, namely: \verb|lookup|, which lookup the node responsible for a
specific hash / ID (for details about Chord, see \citep{Stoica:Chord}).
\newline\newline
Our current implementation fulfills the above interface by utilizing a
centralized server component, this is clearly sub-optimal in what is declared
as a distributed system; below is an argumentation as of why this is the case.

During our project development, we've investigated a few different WebRTC-based
distributed hash table implementations, namely;
\begin{itemize}
    \item WebRTC-explorer (see \citep{diasdavid:webrtc-explorer})
    \item WebRTC-chord (see \citep{diasdavid:webrtc-chord})
    \item Kad-WebRTC (see \citep{kadtools:kad-webrtc})
\end{itemize}
As we found none of these systems, fulfilled our expectations, we ended up 
considering other alternatives; A discussion of these additional alternatives,
is emitted directly after a discussion of the above systems.

\subsection{WebRTC-explorer}
Our experiments with WebRTC-explorer were generally demotivating, the project
does not implement the usual \acs{DHT} interface, but rather a Chord-like
\verb|lookup| network overlay structure interface, as with Chord it's suggested
that one can implement a \acs{DHT} upon this overlay network interface.
\newline
The project is currently in development, by a single masters student; It is our
verdict that the quality, and stability of the project seems to propagate this.

The instability of the project, combined with time-constraints of the project
had us drop this as a possible \acs{DHT} implementation.

\subsubsection{WebRTC-chord}
This project is created as a precursor for WebRTC-explorer, and was through as
a direct implementation of the chord network, using WebRTC. However during 
development the above mentioned masters student found that implementing chord,
requires a lot of (WebRTC) connections, i.e. for successor lists and
finger-tables, and as such he hit the connection limit imposed by the browser;

\begin{displayquote}
    DISCLAIMER: Implementing a full Chord algorithm proved to be an unviable option
    (due to the high number of data-channels that had to be open inside a browser),
    in order to overcome this, I've come up with webrtc-explorer, and adaptable
    Chord like implementation. I'll no longer support this module \\ \medskip
    --- David Dias
\end{displayquote}
Due to this, and the fact that the project has now remained unmaintained for
over two years, we decided against using this as our \acs{DHT} implementation.
\newline\newline
The same developer has previously created a project called \verb|webrtc-ring|
(see \citep{diasdavid:webrtc-ring}), it's deprecated like WebRTC-chord in favor
of WebRTC-explorer, we have therefore not been investigating this project
further.

\subsection{Kad-WebRTC}
Where's the previous two projects are standalone, this is no more than a 
transportation's layer plugin for the Kad project (see \citep{kadtools:kad}).
As such it has the same interface as the Kad project for desktops, which is
indeed the  expected \acs{DHT} interface. The WebRTC transport layer utilizes a
centralized component to setup the WebRTC connections (see
\ref{webrtc-connection-server}), during our experiments this was somewhat easy
to set up and get running, but no interactive example was present in the
project.

We've developed an interactive example for Kad-WebRTC as a part of our
experimentation, and pushed this upstream, such that it is now included in 
the official repository (see \ref{subsec:appendix-kad-webrtc} for details).

While this project did indeed fulfill our expectations, and implements our \acs{DHT}
interface, it is not without it's problems. In truth, we did not expect the 
project to actually run, as the Chord WebRTC implementation mentioned above hits
the limit for WebRTC connections, and since Kademlia generally utilizes a huge
amount of of connections (The official paper on Kademlia, suggests using \acs{UDP}
(see \citep{Maymounkov:Kademlia})).

This sparked our interest, as to how this project could avoid the limit, and as
we investigated this further, we found that this WebRTC transport layer, is
simply used as a plug-in replacement for \acs{UDP}, and as such is connection-less.
Whenever a connection is required, a new one is spawned, this obviously
overcomes the browser limitations on the number of WebRTC connections, but 
instead it establishes new connections constantly.

As previously described WebRTC connections are setup using a centralized server,
and each connection requires in the order of 10 messages total to set up.
Additionally we've found that resynchronization of key-value pairs takes a lot
of time, as Kad is implemented upon the concept of eventual consistency. This
conflicts with our needs, and as such we deemed this project unfeasible for
our choice of \acs{DHT}.

\subsection{\acs{DHT} alternatives}
At this point; we were close to given up our search for a WebRTC \acs{DHT}
implementation, and thus we discussed several alternatives for how to
implement our lookup / searching;
\begin{itemize}
\item Centralized searching
\item Searching in unstructured networks
\item Continue searching for \acs{DHT}s
\end{itemize}

\subsubsection{Centralized searching}
Utilizing a centralized searching component would eliminate the purpose of our
project, as our project would only prove the feasibility of using BitTorrent as
a data transfer technology under a content distribution network, which has been
proven rigorously before, by the common adaptation of BitTorrent.

As such we abandoned this idea again quite quickly.

\subsubsection{Searching in unstructured networks}
We looked into searching in unstructured peer-2-peer networks, there are
several options for implementing this, amongst the first ones is flooding
popularized by the Gnutella network (see \ref{sec:gnutella}).

Flooding does ensure that if a piece of information is contained within the 
network it will be found, however it does not scale well and searching 
generates a lot of traffic (see \citep{Lv:Searching}).
\newline
There are however ways to improve upon flooding, such as described by Gia 
(see \citep{Chawathe:Gia}), where the developers claim to have improved upon
the naive flooding of Gnutella, by three to five orders of magnitude.
They achieve this improvement by the synergy of different modifications,
including flow control, topology adaptation, replication and attention to node
heterogeneity.
\newline\newline
An alternative to flooding, is the $k$-random-walkers algorithm, in which a 
node initiates a search by querying $k$ random neighbours for it, these
neighbours then forward the request to a single of their neighbours, until a 
match is found or a TTL (time to live) variable dries out.

The random-walker approach reduces network usage by two orders of magnitude,
compared to the naive flooding (see \citep{Lv:Searching}), and therefore it is
a lot more scalable, and could be a feasible solution; indeed if it was
combined with replication.
\newline\newline
Due to time constraints we decided against looking further into using an
unstructured peer-2-peer network for searching. In retrospective, we should
have been more open-minded towards different network structures before deciding
upon using a distributed hash table interface.

The Gia paper (see \citep{Chawathe:Gia}) contains an excellent discussion about
unstructured peer-2-peer network vs. structured ones in regards to searching,
the main points being; \acs{DHT}s generally have a harder time in regards to dealing
with churn and handling keyword searches (\acs{DHT}s generally implement exact match
lookup only).

\subsubsection{Continue searching for \acs{DHT}s}
As we rejected the idea of using an unstructured peer-2-peer network for
searching, we decided to continue our search for WebRTC enabled \acs{DHT}s. Which
brought us to BitTorrent \acs{BEP}44 (see \citep{bittorrent:bep44}). In summary, this
protocol extension enables using the tracker-less BitTorrent \acs{DHT} for storing
arbitrary data. While WebTorrent does implement this protocol on desktop (i.e.
in node using \acs{UDP}/\acs{TCP}), it does not implement the protocol using WebRTC, and as
such cannot be used within the context of the web browser, and by extension
within our application.
\newline\newline
Within the given time constraints we have not been able to find a WebRTC enabled
\acs{DHT} implementation, which fulfills our requirements and expectations, and since
this very same time-frame limits us from implementing one ourselves, we've been
forced to stick with our fake implementation.

\subsection{\acs{DHT}-/\acs{PHT}-fake}
As mentioned in the previous section, we've been forced to stick with our faked
implementation, that is we've been forced to use a centralized server component,
to justify this in regards to our hypothesis; we've limited ourselves by the 
following rules;
\begin{itemize}
\item The centralized server component cannot have a richer interface than a 
    normal distributed hash table.
\item Everything our fake does, should be implementable using a WebRTC-based \acs{DHT}.
\end{itemize}
As such, throughout most of our project, we've been relying solely on
exact-match lookup, i.e. as described as a limitation of \acs{DHT}s by the Gia paper
(see \citep{Chawathe:Gia}). We did however stumble upon a paper implementing 
sub-string searching on top of \acs{DHT}s, by implementing a hash-based trie (\acs{PHT})
(see \citep{Ramabhadran:PHT}).

During our continued investigation of the literature within the domain, we 
found another paper, which claims to have improved upon the \acs{PHT} paper, by 
minimizing the maintenance cost of the trie structure by up to 75\%. Their
paper introduces a new data structure called a \ac{LHT},
which alike the \acs{PHT} is built on top of \acs{DHT}s (see \citep{Tang:LHT}).

The publishers of these papers does not release their implementation, and as
such we've not been able to verify their claims, but assuming they do hold the
\acs{PHT} / \acs{LHT} seems fit for our use-case. As such we've expanded our \verb|dht-fake|,
to become a \verb|pht-fake|, i.e. to support sub-string searching.


\section{Browser torrents}
The smart thing about torrents in the browser is that we are not dependent of a server to serve the files, and users does not have to download and install a client to use our solution.

\section{WebTorrent}
WebTorrent is one of the two primary technology used within our project (along 
side with the \acs{DHT}), and as such it deserves it's own treatment, within the 
report.
\newline\newline
WebTorrent is as previously mentioned a full fledged BitTorrent implementation, 
utilizing WebRTC as the underlying transportation layer technology. While it 
does in fact implement both the usual and the WebRTC transportation layers,
we'll only discuss the WebRTC part, as that's what's interesting for our
use-case.
\newline
WebTorrent provides a somewhat simple interface, with the two main methods being;
\begin{itemize}
    \item add; Which takes a \verb|torrentID| (i.e. a torrent file or magnetURI),
        and starts downloading it's content; firing a callback once meta-data has
        been acquired.
    \item seed; Which takes a \verb|File|, and starts seeding it; firing a
        callback once it's seeding.
\end{itemize}
WebTorrent supports steaming of torrent, and this is automatically enabled, if 
one accesses the file of a torrent via. read stream (see \citep{WebTorrent:api}).


\section{Tracker vs. Trackerless}
When running the BitTorrent protocol, we need a bootstrap mechanism, to get in
contact with peers within the network. Several bootstrap mechanisms are exist;
\begin{itemize}
\item Manual peer input
\item Tracker-based peer-lists
\item Trackerless peer-lists
\end{itemize}
The simplest model is that the user provide a number of peers, which are
already within the overlay network. These peers can then by used to further
bootstrap the network (see \citep{bittorrent:bep11}). These pre-known peers
can either be peers connected in the last session (and initially provided by
the software), or found by some means outside the network.

The tracker-based model utilizes a centralized component for acquiring the
initial list of peers to connect to. The flow is simply, a client asks the
tracker for a list of peers given a specific topic (aka. hash), after which the
client is able to contact the peers.

The trackerless model utilizes a \acs{DHT} network for acquiring the initial list of 
peers. The client joins the \acs{DHT} network (by knowing a peer(s) within the network),
and once inside the \acs{DHT} network, it can lookup topics (aka. hashes), and ask
the node responsible for the topic for the peer list.
\newline
- The known peer(s) required to join the \acs{DHT} network, are usually provided by
the \verb|.torrent| file for the download. A reference join node is presented
by BitTorrent, at \url{router.bittorrent.com}, but they recommend running your
own \citep{bittorrent:bep05}.

\subsection{Result}
In our implementation, we've disregarded the manual peer input model, as it is
not user friendly. As such we're left with tracker-based and trackerless, as 
we've limited ourselves to the context of the browser, we're bound to using 
WebRTC, and by extension we need a centralized component to bootstrap these
connections. 

We are however, in theory, free to pick whether we want to use a tracker, or 
utilize a trackerless \acs{DHT} for bootstrapping peers, but only in theory as our
WebRTC-based BitTorrent implementation (as previously mentioned here) does
indeed not support trackerless, and as the provided WebSocket tracker for 
WebTorrent fulfills both the role of bootstrapping WebRTC connections, and the
role of the BitTorrent tracker.

As previously mentioned, had we been able to utilize trackerless, we've been
able to use the trackerless \acs{DHT} as our searching \acs{DHT}, however utilizing a
tracker provides other unique capabilities, namely; centralized control of the
swarm. 
\newline
- A content distributor might like to have this capability, to do data
processing on popular content, or otherwise track the swarm.

While a tracker-based solution utilizes a centralized components, it is not
limited to just using one. Our solution utilizes the 4 default trackers
provided by WebTorrent, but it's trivial to run your own tracker and add it to
the resources in the network.
\newline
- A content provider may want to track how often it's content is downloaded,
and could add the content to the network with just one tracker being their own.


\section{Scraping}
Scraping is a request to the tracker to get information about a torrent, such as how many seeders and leachers it has, how many times it has been downloaded, the status of the tracker (OK or offline), the reason it is offline, etc.
Scraping can be used to find the torrents with bad health and add them to a list. Each peer can then get the list from the \acs{DHT} and seed the torrents on the list.
This can be used to get the health of a torrent i.e. how many seeds it has, and if it at bad health we can add it to a list, and use that list so nodes in the network can download and seed those torrents in the list and help ensure availability of the less popular content.
