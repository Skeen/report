%BitTorrent
BitTorrent is a peer-to-peer protocol for file sharing designed by Bram Cohen in 2001.
The protocol is responsible for around 30 percent of the data uploaded to the internet.
The traditional way to ose the bittorrent protocol is to use a BitTorrent desktop client. That is a computer program that implements the BitTorrent protocol.
The .torrent files comes with a metadata file that includes a trackerlist. The bittorrent tracker is a server that contains information about what peers are interested in a given torrent. The tracker can connect a peer to other peers with the same torrent so the first peer can download and upload torrent data from and to those peers.


%tracker hash forbinde folk 
%bep
BitTorrent enhencement proposals is a place where users of BitTorrent can come with proposals to improve the protocol.
One of these proposals focuses on http seeding \citep{httpSeed}.
This is relevant in our work, because we focus on how we can make streaming work in the browser. The proposal is about changing the metadata file to include a httpseeds key. This key would refer to a list of web adresses where the torrent data can be downloaded from.

\section{Html5}
Html 5 is the latest markup language for writing web applications is was relaesed in 2014. Some of the new tags introduced is video and audio tags wich makes it possiple to play video and audio in the player build into html5.

Html5 also implements local storage. This makes it possible to store content locally in the browsers rather than use cookies, to store the data.
The storage limit it larger than when using cookies (at least 5mb) and the data is stored locally so it does not need to be sent to a server.

\section{browserify}
Browserify is a tool that makes it possible to use modules created to node.js in the browser. It also includes the 'require' functionallity in the browser code. That means code like this:
\begin{verbatim}
var EventEmitter = require('events').EventEmitter;
\end{verbatim}
will make sense in javascript now.

\chapter{Popular music streamers}
The are many music-streamers already reseased and we will let us get inspired by some of the most popular and give a walktrough of them below.

\section{spotify}
Spotify is the most popular online music streamer. The program was released in 2006 and was developed in Sweeden Stockholm by Daniel Ek and Martin Lorentzon. Now their main office is placed in Luxemborg and they have devisions in Stockholm and Göteborg. Their buisness model is that customers can listen to music in exchange for listening to commercials between tracks, or they can pay a monthly fee to be premium members.
The music is placed on servers controlled by spotify in contrast to the peer-to-peer music streamer we will develop.
\begin{figure}[p]
  \centering
  \includegraphics[width=0.9\textwidth]{gfx/Spotify_desktop.jpg}
  \label{spotify_desktop}
\end{figure}
As seen in \ref{spotify_desktop} the music player is placed in the buttom of the application, there is a menu at the left of the screen where your personal playlists can be accessed, and where it is possible to browse for new music and listen to a radio channel.
Spotify now have over 75 mission active users, both from Europe, America and Asia. Spotify both have clients for desktops and mobiles.

\section{Napster}
Napster was when it was released in 1999 a peer-to-peer file sharing service that focussed on audio files. Napster was co-founded by Shawn Fanning, John Fanning, and Sean Parker The Napster site was only operational untill 2001 because it was sued by Metallica for sharing music illigally. It was then brought down by court order. Metallica learned that their song "I dissapear" was avaiable on Napster before it was released witch also led to it be played on radios. On March 13, 2000 Metallica filed a lawsuit against Napster. In addition to the Napster weppage desktop client was develloped both to windows and later to MAC OS. 
