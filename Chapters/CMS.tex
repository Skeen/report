\section{Content Management System}
As previously stated we utilize a \acs{DHT} overlay network for searching. While it's
currently faked, we do still add content as we would assuming we were utilizing
a non-faked implementation.

Whenever a song is uploaded, the following take place:
\begin{itemize}
\item We parse out the meta-data, and create a Song object from it.
\item We add the song to the \acs{DHT}, with the key being the sha1 hash of it's title.
\item We check if the songs album is already within the \acs{DHT};
\begin{itemize}
\item If it is, we check if the artists are in correspondance, i.e. between
    this song, and the \acs{DHT} content, if it is not, we update the information.
    Additionally, we add this song to the list of songs for the album.
\item If it is not, we add a new entry in the DHt, with the artists of this
    song, and add this song as the only song of the album.
\end{itemize}
\item We check if each of the artist(s) from the song, exists in the \acs{DHT}.
\begin{itemize}
\item If they do, we update their album information (if necessary)
\item If they do not, we add them to the \acs{DHT}.
\end{itemize}
\end{itemize}
In the above, we say that we add a X to Y, say a song to an album. The actual 
operation which takes place, is that we pull in the current \acs{JSON} representation
of the album, then we add the song's title and it's hash (key into the \acs{DHT}),
before pushing the modified \acs{JSON} representation back into \acs{DHT}.

In the faked implementation this is somewhat easy, while in a real \acs{DHT} one
would have to implement a smarter updating scheme, i.e. to avoid race conditions.
\newline
- A common scheme for databases is the transactional pattern, in which an update
either completely succeeds or fails without side-effects. We have not experimented
with implementing this sort of feature on top of a \acs{DHT} network, but we envision
that a sub-protocol could be executed by the involved peers (i.e. the peers
responsible for the song, album and artists respectively).

Additionally the network would have to be protected against malicious updates.
We deem this outside the scope of our implementation, and as such we do not
provide a solution to this, but point towards assymmetric cryptography for a 
solutions.

A disclaimer has to be made; we (as previosly mentioned) currently fake the 
prefix hash trie implementation, as such we actually do some lookups which
bypasses the \acs{DHT} interface. In order to utilize the same interface, we've
prefixed all sha1 hashes in the network with 'sha1:', and all queries that does
not have this prefix are supposed to be implemented on the prefix hash trie.
