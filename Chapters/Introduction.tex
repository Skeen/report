%% Teaser
In this project we propose a new approach to content distribution networks,
namely an approach which is fully decentralized and utilizes common in-place 
software infrastructure.
\newline\newline
%% Disclaimer
We've decided to limit the scope of our study-case from general content
distribution networks to multimedia content distribution networks
(specifically music distribution networks).
\newline
% Problem definition
In particular the case we want to look into, is sharing multimedia content from
an established content distribution network to a peer, which requests a
specific resource from the network.
\newline\newline
% Related solutions
Common in-place solutions to this problem, utilizes a classic client-server
architecture, in which a centralized distributor offers a distribution service,
to a set of peers, which usually pay for the provided service, either directly
or via. ad revenues.

Examples of commmon solutions, which utilize this traditional model, includes;
Spotify, Apple Music, SoundCloud, Netflix and YouTube.
\newline
These services and the systems supporting them, are well implemented and
fully functional. There are however challenges in regards distributing
multimedia content using the centralized client-server model, namely the 
astronomical amounts of data-bandwith and storage required to do so.
\newline
It is estimated that multimedia content distribution accounts for over 50\% of
the global Internet traffic, with the largest provider (Netflix) accounting for
almost 40\% itself\citep{sandvine:2015}.
\newline
Distributing such massive amounts of data in the client-server model setting, 
implicates enormous servers implemented via. globally distributed data-centers.
\newline\newline
% Hypothesis
We hypothesize that it is feasible to implement a completely distributed
content distribution network, using modern components, and that such a system
would reap of fruits of distributed systems, such as high scalability, 
performance, availability and robustness, while retaining the properties of 
centralized systems, such as indexing and control over the distribution network.
\newline
That is we hypothesize that we can combine the content indexing and control of
centralized distribution networks, with a distributed data sharing network,
such as BitTorrent, and thus produce a system with the qualities of both the
centralized and the distributed systems.
\newline\newline
% Test
To test our hypothesis, we'll build a miniture version of the aforementioned
system, with some concept of distributed indexing and distributed data
sharing.
\newline
Then we'll test if the system is usable, and feasible in regards to the
mentioned performance metrics, i.e. that it distributes work-load across the
system, that it's highly scalable, performant, available, ect.
\newline\newline
% References
%% TODO: Fix references to use \ref
Chapter 2, discusses related work, i.e. related systems, technologies and 
results. In Chapter 3 we sum up these, and argue why some systems may have
failed and why some systems have succeded. Additionally describe in greater
detail the systems upon which our system is implemented, and argue their
usefulness to our cause. Chapter 4 sketches out the design of our system, both 
the implemented design and the full fledged design of a complete system.
Chapter 5 goes on to describe the extend of our implemented system, along with
some of the core challenges faced during development.
Appendix~\ref{sec:appendix-contributions} includes the list of contributions
for the implementation, while Part~\ref{part:demo} includes a manual and
walkthrough demo of the implementation. For completeness we have included
our accepted project proposal as Appendix~\ref{sec:appendix-project}.
Chapter 6 sums up the results of our system tests, on which our conclusion in
Chapter 7 is based upon.

%% TODO: Motivation, well documented case problem
