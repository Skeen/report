% WebTorrent

% Songs and Metadata
Adding support for music file types not already supported by the browser 
would be an major undertaking, 
and would not contribute much towards the goals of this project, 
so we have decided to make use of existing browser capabilities.
The supported music filetypes differ greatly from eachother internally; 
their audio codings can be different, their metadata can be different,
and even the way they are rendered and treated by the browsers can differ.
To make matters even worse, 
creators of music files are free to exclude or falsify metadata, and can even leave it out entirely.
Clearly, some kind of common abstraction is needed to encapsulate the uncertain terrain of music.

% Audio Tag and limitations
Modern browsers already support native playback of audio files through the Audio HTML5 tag, 

% LocalStorage vs LocalForage
To have persistency in user sessions, we needed a way of storing data to the disk. There are several standard ways to store data in web browsers, but all of these have their own drawbacks.

Traditional browser cookies can store up to 4KB of data, 
not enough to save even one song. 
To store a song in cookies, 
we would need to create thousands of cookies, 
and store parts of a song in each one, 
this seemed like a difficult and inelegant solution.

Seeing the need for larger storage spaces, 
industry concerns have added LocalStorage to the HTML 5 standard,
which allows websites to store arbitrarily large amounts of data,
but this again suffers significant drawbacks: 
implementation varies greatly between browsers, 
and can in some cases be wiped after reaching only 5 megabytes, 
hardly enough for our purposes.

Some browsers have also added support for database storage, via IndexedDB and mysql, but this is not supported in every browser.

To overcome these challenges, we use a library called localForage, 
which checks what the host browser supports,
and makes use of the best available storage method.
We found through experiments,
that localForage has issues saving javascript objects to disk when these objects contain large binary blobs, 
these could consume as much as 4 gigabytes of space.
Seperating the generated metadata object and song binary before placing the in storage, 
seemed to solve this issue.

% Granularity of torrents
