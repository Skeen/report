\section{The optimal solution}
As previously stated in the introduction we are trying to build a decentrilized version of Spotify.
This means that the system should be a music-streamer with an interface that runs in a browser,
and makes it possible for the users to search for a song.

All the users (peers) should be the ones to provide the content of the system.
We strove to achive a minimum implementation of spotify.
Spotify have a social aspect where you can login, save your playlist, see your friends activity and discover new music.
Spotify have ordered their music into genres and top charts.
In Spotify there is a radio function where music is played based on artist or genres.
A radio function would be a good function in our system, because it would make users want to discover new music.
Our system could have a navigation pane in the left side,
 where it could be possible to see what music your friends are listening to,
  browse the top 50 charts with mixed music and for a given genre.
Users could be given suggestions based on their recently played music.

\section{Planned System}
The system should support music file sharing using WebTorrent,
searching for content in a DHT system,
and playback of music in the browser.
We plan the following features:
A radio function would be a good function in our system, because it would make users want to discover new music.
\begin{itemize}
	\item Users can add music to the network
	\item Users can search for content, artists and albums
	\item Users can get music from peers
	\item Users can see ongoing downloads
	\item Users can see seeding content
	\item Users can see local content
	\item Users can play local content
	\item Users can play downloading content
	\item Peers retrieve and seed bad health content
	\item Peers use intelligent downloading strategy
	\item Peers persistantly store content
	\item Peers retrieve content from storage
	\item Peers seed local content
\end{itemize}

Searching for content should be handled by a Content Management System,
which should be a \acs{DHT} if one is available.
The \acs{GUI} has to be impressive enough to show that our system
could feasibly run behind a large, real-life, popular music service.

With the time limits of a quarter year,
we have limited the scope of our system compared to Spotify,
particularely in the areas of the \acs{GUI}, and development of subprojects.
In case we cannot find a reliable, functioning \acs{DHT} system developed for WebRTC,
we will instead make use of a faked \acs{DHT}.
