% **************************************************************************************************************
% A Classic Thesis Style
% An Homage to The Elements of Typographic Style
%
% Copyright (C) 2015 André Miede http://www.miede.de
%
% If you like the style then I would appreciate a postcard. My address 
% can be found in the file ClassicThesis.pdf. A collection of the 
% postcards I received so far is available online at 
% http://postcards.miede.de
%
% License:
% This program is free software; you can redistribute it and/or modify
% it under the terms of the GNU General Public License as published by
% the Free Software Foundation; either version 2 of the License, or
% (at your option) any later version.
%
% This program is distributed in the hope that it will be useful,
% but WITHOUT ANY WARRANTY; without even the implied warranty of
% MERCHANTABILITY or FITNESS FOR A PARTICULAR PURPOSE.  See the
% GNU General Public License for more details.
%
% You should have received a copy of the GNU General Public License
% along with this program; see the file COPYING.  If not, write to
% the Free Software Foundation, Inc., 59 Temple Place - Suite 330,
% Boston, MA 02111-1307, USA.
%
% **************************************************************************************************************
\RequirePackage{fix-cm} % fix some latex issues see: http://texdoc.net/texmf-dist/doc/latex/base/fixltx2e.pdf
\documentclass[ twoside,openright,titlepage,numbers=noenddot,headinclude,%1headlines,% letterpaper a4paper
                footinclude=true,cleardoublepage=empty,abstractoff, % <--- obsolete, remove (todo)
                BCOR=5mm,paper=a4,fontsize=11pt,%11pt,a4paper,%
                ngerman,american,%
                ]{scrreprt}

%********************************************************************
% Note: Make all your adjustments in here
%*******************************************************
\input{classicthesis-config}

%********************************************************************
% Bibliographies
%*******************************************************
\addbibresource{Bibliography.bib}
\makeatletter
\def\blx@maxline{77}
\makeatother
%********************************************************************
% Hyphenation
%*******************************************************
%\hyphenation{put special hyphenation here}

\begin{document}
\frenchspacing
\raggedbottom
\selectlanguage{american} % american ngerman
%\renewcommand*{\bibname}{new name}
%\setbibpreamble{}
\pagenumbering{roman}
\pagestyle{plain}
%********************************************************************
% Frontmatter
%*******************************************************
%*******************************************************
% Little Dirty Titlepage
%*******************************************************
\thispagestyle{empty}
%\pdfbookmark[1]{Titel}{title}
%*******************************************************
\vspace*{\fill}\noindent{%\rule{\linewidth}{1mm}\\[4ex]
{\Large \spacedallcaps \myTitle\\}\\[1ex]
{\large \spacedallcaps \mySubtitle}\\[2ex]
{\Large \spacedlowsmallcaps \myName, \myStudentId }\\[2ex]
\noindent\rule{\linewidth}{1mm}\\[4ex]
\noindent{\large \spacedlowsmallcaps \myDegree\\[1ex] 
\monthname\ \the\year  \\[1ex] Advisor: \myProf \\[23ex]}\\[\fill]}
\includegraphics[width=\linewidth]{gfx/logo}\clearpage


\include{FrontBackmatter/Titlepage}
\thispagestyle{empty}

\hfill

\vfill

\noindent\myNameOne, \myNameTwo, \myNameThree: \textit{\myTitle,} \myDegree,
\textcopyright\ \myTime

%\bigskip
%
%\noindent\spacedlowsmallcaps{Supervisors}: \\
%\myProf \\
%\myOtherProf \\ 
%\mySupervisor
%
%\medskip
%
%\noindent\spacedlowsmallcaps{Location}: \\
%\myLocation
%
%\medskip
%
%\noindent\spacedlowsmallcaps{Time Frame}: \\
%\myTime

\cleardoublepage%*******************************************************
% Dedication
%*******************************************************
\thispagestyle{empty}
%\phantomsection 
\refstepcounter{dummy}
\pdfbookmark[1]{Dedication}{Dedication}

\vspace*{3cm}

\begin{center}
    Sharing is good, and with digital technology, sharing is easy. \\ \medskip
    --- Richard Stallman
\end{center}

\medskip

\hfill

\vfill

\noindent\myNameOne, \myNameTwo, \myNameThree: \textit{\myTitle,} \myDegree,
\textcopyright\ \myTime



%\cleardoublepage\include{FrontBackmatter/Foreword}
\cleardoublepage%*******************************************************
% Abstract
%*******************************************************
%\renewcommand{\abstractname}{Abstract}
\pdfbookmark[1]{Abstract}{Abstract}
\begingroup
\let\clearpage\relax
\let\cleardoublepage\relax
\let\cleardoublepage\relax

\chapter*{Abstract}
Most mainstream multimedia content distribution networks and system, utilize
the traditional client-server architecture; as such scalability is by means of
larger and larger global datacenters, with ever increasing operation costs.
We've set forth the hypothesis that a fully decentralized multimedia content 
distribution network, with the properties of centralized ones is both feasible
and achiveable using modern software components.
Additionally we've limited ourselves to the somewhat limited execution
environment presented by the context of modern web-browsers, as we believe this
is the multimedia content distribution platform of the future.
We've developed a decentralized system for this platform, which utilizes a WebRTC
enabled BitTorrent implementation for low-level data transmission, and a 
(currently) faked prefix hash-trie for content indexing and searching.
We found that no proper prefix hash-trie (or even distributed hash table)
implementations which utilize WebRTC exists, and as such; we've been forced to
utilize a faked component for this (faked by centralization). As a result,
we're unable to utilize the system to prove our hypothesis, and we can only
provide our system as evidence towards the validity of it.

\vfill

\pdfbookmark[1]{Sammenfatning}{Sammenfatning}
\chapter*{Sammenfatning}
De fleste populære multimedie indholdsfordelingssystemer og netværk,
benytter den traditionelle klient-server arkitektur, hvormed systemet
udelukkende kan skaleres ved at bygge større og større globale data-centre
med evigt stigende udgifter til følge.
Vi har fremsat hypotesen at et fuldkomment decentraliseret multimedia 
indholdsfordelingssystem med egenskaberne fra et centraliseret system er både
realistisk og opnåeligt ved brug af moderne software komponenter.
Vi har begrænset os selv til at vores kode skal eksekveres i en moderne
browser, da vi vurdere dette som fremtidens multimedia indholdsfordelings
platform.
Vi har udviklet et decentraliseret system for denne platform. Systemet benytter
en BitTorrent implementering med et WebRTC transportlag for alle data
transmissioner, samt et (i øjeblikket forloren) prefix hash-trie for
indeksering og søgning efter indhold.
Da vi ikke har været i stand til at finde en ordenlig prefix hash-trie (eller
blot distributeret hash tabel) implementering, som benytter WebRTC har vi været
nødsaget til at benytte vores centralisered fake komponent. Dette betyder 
imidlertid at vi ikke kan benytte systemet direkte til at bekræfte vores 
hypotese, men må nøjes med at fremstille systemet som evidens for korrektheden
af denne.

\endgroup			

\vfill

%\cleardoublepage\include{FrontBackmatter/Publications}
\cleardoublepage\include{FrontBackmatter/Acknowledgments}
\pagestyle{scrheadings}
\cleardoublepage\include{FrontBackmatter/Contents}
%********************************************************************
% Mainmatter
%*******************************************************
\cleardoublepage\pagenumbering{arabic}
%\setcounter{page}{90}
% use \cleardoublepage here to avoid problems with pdfbookmark
\cleardoublepage
% What should a thesis contain:

\ctparttext{The following reflects what I believe to be a good structure
  for a report or a thesis in experimental computer science. 
It contains a natural progression from the general to the specific, and
from the work of others to the work by the authors, each chapter
forming the foundation of the next.}
\part{The Proper Structure of a Thesis}
\label{part:prop-struct-thes}
\cleardoublepage

\chapter{Introduction}
\label{ch:introduction}
%% Teaser
In this project we propose a new approach to content distribution networks,
namely an approach which is fully decentralized and utilizes common in-place 
software infrastructure.
\newline\newline
%% Disclaimer
We've decided to limit the scope of our study-case from general content
distribution networks to multimedia content distribution networks
(specifically music distribution networks).
\newline
% Problem definition
In particular the case we want to look into, is sharing multimedia content from
an established content distribution network to a peer, which requests a
specific resource from the network.
\newline\newline
% Related solutions
Common in-place solutions to this problem, utilizes a classic client-server
architecture, in which a centralized distributor offers a distribution service,
to a set of peers, which usually pay for the provided service, either directly
or via. ad revenues.

Examples of commmon solutions, which utilize this traditional model, includes;
Spotify, Apple Music, SoundCloud, Netflix and YouTube.
\newline
These services and the systems supporting them, are well implemented and
fully functional. There are however challenges in regards distributing
multimedia content using the centralized client-server model, namely the 
astronomical amounts of data-bandwith and storage required to do so.
\newline
It is estimated that multimedia content distribution accounts for over 50\% of
the global Internet traffic, with the largest provider (Netflix) accounting for
almost 40\% itself\citep{sandvine:2015}.
\newline
Distributing such massive amounts of data in the client-server model setting, 
implicates enormous servers implemented via. globally distributed data-centers.
\newline\newline
% Hypothesis
We hypothesize that it is feasible to implement a completely distributed
content distribution network, using modern components, and that such a system
would reap of fruits of distributed systems, such as high scalability, 
performance, availability and robustness, while retaining the properties of 
centralized systems, such as indexing and control over the distribution network.
\newline
That is we hypothesize that we can combine the content indexing and control of
centralized distribution networks, with a distributed data sharing network,
such as BitTorrent, and thus produce a system with the qualities of both the
centralized and the distributed systems.
\newline\newline
% Test
To test our hypothesis, we'll build a miniture version of the aforementioned
system, with some concept of distributed indexing and distributed data
sharing.
\newline
Then we'll test if the system is usable, and feasible in regards to the
mentioned performance metrics, i.e. that it distributes work-load across the
system, that it's highly scalable, performant, available, ect.
\newline\newline
% References
%% TODO: Fix references to use \ref
Chapter 2, discusses related work, i.e. related systems, technologies and 
results. In Chapter 3 we sum up these, and argue why some systems may have
failed and why some systems have succeded. Additionally describe in greater
detail the systems upon which our system is implemented, and argue their
usefulness to our cause. Chapter 4 sketches out the design of our system, both 
the implemented design and the full fledged design of a complete system.
Chapter 5 goes on to describe the extend of our implemented system, along with
some of the core challenges faced during development.
Appendix~\ref{sec:appendix-contributions} includes the list of contributions
for the implementation, while Part~\ref{part:demo} includes a manual and
walkthrough demo of the implementation. For completeness we have included
our accepted project proposal as Appendix~\ref{sec:appendix-project}.
Chapter 6 sums up the results of our system tests, on which our conclusion in
Chapter 7 is based upon.

%% TODO: Motivation, well documented case problem


%Whereas the purpose of the Introduction chapter was to entice and
%convince the reader that work reported is interesting, that the author
%is asking the right questions about it, and reading about it will be
%worthwhile, the purpose of the Related Work chapter is to
%demonstrate that the author possesses a fine overview and keen
%understanding of the topic of the work.  Note that while the title of
%the chapter is ``Related Work'', it might as well be called
%``\emph{Relevant} Work'' in that you should only include work that are
%useful or relevant to your purpose. 
%
%Writing about others' work can be challenging---it is easy to succumb
%to just writing condensed summaries, which is just as tedious to read
%as they are to write. A better method is to gain an overview over the
%field of inquiry, and then establish in the first section what aspects
%or dimensions are crucial to systems or methodologies such as the ones
%described. This demonstrates to the reader that the author has
%understanding and judgement. Having done this, every paper or work can
%then be described in those established terms. This makes for easier
%and much more structured writing, and it also helps the reader
%differentiate the systems and works reported on. If there are multiple
%works that cover approximately the same area (\eg, using the same
%technique), you may mention several, but only go into detail with the
%most significant or representative one.
%
%The chapter can then be concluded with a table summarising all the
%work reported on using the aspects defined in the introduction of the
%chapter.
%
%A crucial element of this chapter is that it concerns the work of
%others and \emph{only} that. While the selection of aspects or
%dimensions described above invariantly will reflect your own focus,
%that should be the extend of which your own work and plans influence
%this chapter.  Your own judgement comes in the next chapter.

%\section{Frameworks and Technologies}
%\label{sec:fram-techn}

\chapter{Related Work}
\label{cha:related-work}
%BitTorrent
\section{bittorrent}
BitTorrent \citep{bittorrent:bep03} is a peer-to-peer protocol for file sharing designed by Bram Cohen in 2001.
The protocol is responsible for around 30 percent of the data uploaded to the internet.
The traditional way to ose the bittorrent protocol is to use a BitTorrent desktop client. That is a computer program that implements the BitTorrent protocol.
The .torrent files comes with a metadata file that includes a trackerlist. The bittorrent tracker is a server that contains information about what peers are interested in a given torrent. The tracker can connect a peer to other peers with the same torrent so the first peer can download and upload torrent data from and to those peers.

\section{File sharing}
Torrents is a very popular and easy method to share large files, because every user providing bandwidth i.e. the seeding of the torrents.
The payload of the server that initially shared the file does not have to be big, because the user will share the file with other users instead of overloading the server.
%tracker hash forbinde folk 
%bep
BitTorrent enhencement proposals is a place where users of BitTorrent can come with proposals to improve the protocol.
One of these proposals focuses on http seeding \citep{bittorrent:bep17}.
This is relevant in our work, because we focus on how we can make streaming work in the browser. The proposal is about changing the metadata file to include a httpseeds key. This key would refer to a list of web adresses where the torrent data can be downloaded from.

\section{Html5}
Html 5 is the latest markup language for writing web applications is was relaesed in 2014. Some of the new tags introduced is video and audio tags wich makes it possiple to play video and audio in the player build into html5.

Html5 also implements local storage. This makes it possible to store content locally in the browsers rather than use cookies, to store the data.
The storage limit it larger than when using cookies (at least 5mb) and the data is stored locally so it does not need to be sent to a server.

\section{WebRTC}
WebRTC is an API drafted by W3C that supports browser-to-browser connections,
which focuses on peer-to-peer connections instead of traditional
client-server networks like WebSockets, AJAx and Server Sent Events.
Traditional networks suffer from high latencies and require dedicated servers to do the heavy lifting, 
as browsers cannot listen for connections with WebSockets.

Before WebRTC, establishing direct peer-to-peer connections in the browser required
additional plugins in the browser such as Flash or Java, which would then download and
execute the networking code,
these plugins did not come with most browsers,
which means many users did not have them available.

% TODO: write about peerconnection requiring central server to establish connection
\label{webrtc-connection-server}
When using WebRTC, we must first notify the remote peer of our intention
to open a peer-to-peer connection, so it can start listening for incoming packet,
we also have to establish the necessary routing paths to each other peer on both sides,
and relay this information,
and finally establish the intended parameters: protocols, encoding used, and so on.
Browser clients establish a WebSocket connection the server, 
asks to have peer-to-peer connections established,
and then receives established connections to other peers.
As the server only establised the peer-to-peer connection, 
it does not have to bear much workload on its perhaps limited resources,
and is much less likely to become a bottleneck of the system.
Peers can continue their connection even if the server is shut down later.

\section{Browserify}
Browserify is a tool that makes it possible to use modules created to node.js in the browser. It also includes the 'require' functionallity in the browser code. That means code like this:
%\begin{verbatim}
%var EventEmitter = require('events').EventEmitter;
%\end{verbatim}
will make sense in javascript now.

\section{Popular music streamers}
The are many music-streamers already reseased and we will let us get inspired by some of the most popular and give a walktrough of them below.

\subsection{spotify}
Spotify is the most popular online music streamer. The program was released in 2006 and was developed in Sweeden Stockholm by Daniel Ek and Martin Lorentzon. Now their main office is placed in Luxemborg and they have devisions in Stockholm and Göteborg. Their buisness model is that customers can listen to music in exchange for listening to commercials between tracks, or they can pay a monthly fee to be premium members.
The music is placed on servers controlled by spotify in contrast to the peer-to-peer music streamer we will develop.
\begin{figure}[p]
  \centering
    \includegraphics[width=0.9\textwidth]{gfx/Spotify_desktop.jpg}
  \caption{A picture of the GUI in the Spotify desktop application}
  \label{fig:spotify}
\end{figure}
As seen in \ref{fig:spotify} the music player is placed in the buttom of the application, there is a menu at the left of the screen where your personal playlists can be accessed, and where it is possible to browse for new music and listen to a radio channel.
Spotify now have over 75 mission active users, both from Europe, America and Asia. Spotify both have clients for desktops and mobiles.

\subsection{Napster}
Napster was when it was released in 1999 a peer-to-peer file sharing service that focussed on audio files. Napster was co-founded by Shawn Fanning, John Fanning, and Sean Parker The Napster site was only operational untill 2001 because it was sued by Metallica for sharing music illigally. It was then brought down by court order. Metallica learned that their song "I dissapear" was avaiable on Napster before it was released witch also led to it be played on radios. Napster peaked at 25 million users in january 2001 On March 13, 2000 Metallica filed a lawsuit against Napster. Many  In addition to the Napster weppage, a desktop client was developed both to windows and later to MAC OS.
Napster had to pay 36\$ in royalties to various music companies and had to declare itselves bankrupt. Roxio bought the assets of Napster to relaunch it as a online music store, where users had to pay money for each track. In may 2006 Roxio launched a free verion of Napster where users where able to stream full length tracks, where the service was powered by adds. They had the limit that each song could only be streamed 3 times each by every user, but they had 8million songs to choose from.
The free Napster service was discontinued in March 2010, because Napster was sold to Best Buy in January 2010.
Rapsody (another streaming and download service) bought Napster in 2011 and transformed it into a subscription based streaming client for desktop or phones.

\subsection{popcorn-time}
Popcorn-time is a multi-platform bittorrent client for streaming videos. The popcorn-time interface is much simillar to that of Netflix. It presents the user with thumbnail immages of the movies and when the user selects a picture the movie is then downloaded with the bittorrent protocol, and played in the build-in player.
 When a download starts the movie is also seeded to other peers in the network. The seeding continues until the content i deleated with usually happens when the application closes.
 In 2014 developers made popcorn-time available on android, and support for Chromecast and Apple TV.

\section{Netflix}
Netflix is the largest streaming company. As seen in \citep{netflix} they were looking into the possibilities of using webtorrent in their company. At the time of the article Netflix had a job application for a peer-to-peer senior developer. Netflix is not using any peer-to-peer protocol today so the project was not a huge success.


%
\section{Browser versus desktop}
The main idea behind making a streamer for webbrowsers is, that it provides an easy way for the users to use the application without first having to install a client.


%Related work need not be only published academic work. In many cases,
%it is also relevant to describe crucial frameworks and technologies
%that will be used or are relevant for the thesis.  This does not mean
%that all employed technologies should be described in detail, but
%frameworks and technologies that are unusual (for lack of a better
%word) could be described here. \Eg, there is no need to describe an
%ordinary network stack, but if the work involves GPU programming, a
%description of the chosen architecture might well be relevant, as it
%informs all the following chapters.

\chapter{Analysis}
\label{cha:analysis}
% Audio Tag capabilities
\section{HTML5 Audio Tag}
\label{sec:HTML5_audio_tag}

Adding support for music file types not already supported by the browser 
would be an major undertaking, 
and would not contribute much towards the goals of this project, 
so we have decided to make use of existing browser capabilities.
\newline\newline
Most modern browsers support the \acs{HTML}5 standard Audio tag,
also called \acs{HTML}5 Audio and WebAudio,
which presents a simple \acs{GUI} and audio playback to the user.
Browsers also support a similar Video tag, 
which could also be used for audio playback purposes.
The audio and video tags support both files and streams, 
allowing playback while a file is still being downloaded.
\newline
Supported audio codings differ greatly between browsers:
Firefox and Opera do not support the proprietary formats Mp3, Mp4 or ADTS
without additional software on the machine.
Safari does not support the free open source formats WebM or Ogg,
and Internet Explorer and Microsoft Edge does not support anything other than Mp3 and Mp4.
Google chrome notably supports all the above formats.
\newline

%\includegraphics{gfx/audioTag}
\begin{figure}[h]
    \centering
    \includegraphics[scale=0.5]{gfx/audioTag.jpg}
    \caption{A picture of Audio Tag \acs{GUI} in Firefox}
    \label{fig:audiotag}
\end{figure}

The audio tag itself offers a very simplistic yet very inflexible \acs{GUI} to the user, 
it includes the usual progress bar, volume, play+pause buttons in its interface, 
these elements cannot be easily changed, but can be disabled and hidden entirely, 
which lets the audio playback be controlled by JavaScript code instead.

% Song Information
\section{Music File-types}
The supported music file types differ greatly from each other internally; 
their audio codings can be different, their meta-data can be different,
and even the way they are rendered and treated by the browsers can differ.
To make matters even worse creators of music files are free to exclude or falsify meta-data, 
and can even leave it out entirely.
Clearly, some kind of common abstraction is needed to encapsulate the uncertain terrain of music files.

\section{WebRTC limitations}
% WebRTC limitations
WebTorrent relies heavily on WebRTC and particularly on WebRTCs dataChannels.
DataChannels are not supported in the Internet Explorer and Safari browsers, 
so these cannot support WebTorrent either.

Creating a data channel is a performance intensive task:
Peers have to contact a signaling service to inform a remote peer of its intention
to make a peer-to-peer connection, both peers have to perform NAT traversal to establish routing information 
and create necessary firewall arrangements, share their routing information,
and finally agree on protocols, encodings and so on.
RTC datachannels have a large overhead when being established,
but can perform low latency communications once established, 
thus we would like to retain datachannels for as long as possible to avoid this overhead.
Many \acs{DHT}, like Kademlia, will establish and discard connections frequently,
and might suffer major overhead when implemented in WebRTC over the datachannel.`

% LocalStorage
\section{Storage}
Ensuring availability of content and robustness in a BitTorrent network requires persistent offline data storage:
without persistent data storage, if all users were to leave the network and return later, all content on the 
network would be lost.
\newline\newline
Until recently the only way to have data persistency through browsers has been cookies, 
which allows saving up to 4KB in each cookie, and can hold at least 50 cookies per domain, 
which allows for a total of 200KB of guaranteed available storage, possibly more depending on the browser.
\newline\newline
Seeing the need for larger amounts of persistent data, browsers now implement Web Storage 
(also known as DOM storage), a new standard by W3C (see \citep{WebStorage})
which consists of sessionStorage and localStorage.

Mozilla also defines a globalStorage for their Firefox browser, 
but this is not supported by any other major browsers, so we will disregard it for this project.

SessionStorage is intended to be wiped whenever the current session ends, 
even if the user agent has not requested it, making it also unsuited for this project.

Data stored using LocalStorage is only wiped when explicitly requested by the user agent or 
by some security policy specified by the user agent beforehand.
\newline\newline
The current Web Storage specification draft is somewhat ambiguous, 
it suggests that browsers implement support for at least 5MB of storage of each kind per domain,
but the actual support varies wildly even between versions of the same browser:

\begin{table}[h]
	\centering
	\begin{tabular}{l | r}
        Browser   & Storage limit \\ \hline
		Chrome 18 & No limit  \\
		Chrome 19 & 1    MB   \\
		Chrome 36 & 5    MB   \\
		Chrome 46 & 4.75 MB   \\
		Firefox   & 5    MB   \\
		IE        & 10   MB   \\
		Safari    & 5    MB   \\
	\end{tabular}
	\caption{Table of size limits of localstorage}
	\label{table:browserls}
\end{table}

Additionally, Firefox allows the user to specify any LocalStorage maximum size, even zero. 
Google have attempted to implement a quota \acs{API}, 
which allows websites to ask user agents to increase localstorage limitations,
this has not been adopted by the other browsers, so we had to look for something else.
\newline\newline
In addition to the Web Storage standard, W3C is also working on a standard for IndexedDB in browsers, 
which allows browser-side JavaScript to access a local database on the host machine.

It should be noted that WebTorrent, 
an essential library in our project, 
only officially supports Firefox and Chrome 
both of which do support IndexedDB.

\section{BEP - BitTorrent Enhancement Proposals}
%bep
BitTorrent enhancement proposals is a place where users of BitTorrent can come with proposals to improve the protocol.
One of these proposals focuses on \acs{HTTP} seeding \citep{bittorrent:bep17}.
This is relevant in our work, because we focus on how we can make streaming work in the browser. The proposal is about changing the meta-data file to include a \acs{HTTP} seeds key. This key would refer to a list of web addresses where the torrent data can be downloaded from.
If support for \acs{HTTP} seeds would be added to the BitTorrent protocol it would mean that there is more uptime of seeds because the ftp and \acs{HTTP} server would always be up.

In another proposal \citep{bittorrent:bep09} it is proposed to make it possible for clients to join a swarm and complete a download without having to download a .torrent file first, from an info hash contained in magnet links. The gains from using magnet links over torrents are that information about the torrent can be downloaded directly from other peers. Regular torrent files needs to be placed upon a web-server where it takes up space. The server must be up and running for the users to get information about the torrent, and that is not needed with using magnet links.
Now users can just share a link instead of sharing a file on a server.

Private torrents is another feature in BitTorrent \citep{bittorrent:bep27}. They idea is that users can define a torrent to be private by setting 'private=1' in the meta-info (.torrent) file. Private trackers can be used to control what users are allowed to download a torrent. The tracker is a server with a list of the peers that contain pieces of the torrent and a list of users allowed to get the specific torrent.
The tracker will refuse to provide a seed-list to those not allowed to get that torrent.

%Storing arbitrary data in the \acs{DHT}
In a proposal \citep{bittorrent:bep44} it is suggested that the \acs{DHT} of BitTorrent is extended with the possibility of storing arbitrary data instead of just storing key-value pairs with hashes and associated \acs{IP}-addresses.
With this change the \acs{DHT} could be used to store information about the torrent e.g. song length, author, file size, etc.

%Why browser torrents
\section{Browser torrents}
The smart thing about torrents in the browser is that we are not dependent of a server to serve the files, and users does not have to download and install a client to use our solution.

%Scraping
\section{Scraping}
Scraping is a request to the tracker to get information about a torrent, such as how many seeders and leachers it has, how many times it has been downloaded, the status of the tracker (OK or offline), the reason it is offline, etc.
Scraping can be used to find the torrents with bad health and add them to a list. Each peer can then get the list from the \acs{DHT} and seed the torrents on the list.
This can be used to get the health of a torrent i.e. how many seeds it has, and if it at bad health we can add it to a list, and use that list so nodes in the network can download and seed those torrents in the list and help ensure availability of the less popular content.


%% TODO: Move this about
% TODO: Was it html5?
WebSockets are a transport layer technology introduced in HTML5, it enables BSD
socket-like data transfers between the browser and a web-socket enabled backend
server. That is; a single open bi-directional connection, over which multiple
messages may be transmitted.

% TODO: How does it work?
A WebSocket connection is intialized by a HTTP request, bla bla bla.

% TODO: more

%% TODO: Move this about
\section{WebTorrent}

WebTorrent is one of the two primary technology used within our project (along 
side with the DHT), and as such it deserves it's own treatment, within the 
report.

WebTorrent is as previously mented a full fledged BitTorrent implementation, 
utilizing WebRTC as the underlying transportation layer techonlogy. While it 
does in fact implement both the usual and the WebRTC transportation layers,
we'll only discuss the WebRTC part, as that's what's interresting for our
use-case.

WebTorrent provides a somewhat simple interface, with the main methods being;
% TODO: Look-up and write
\begin{itemize}
\item leech
\item seed
\end{itemize}

% TODO: Information about Blobs, file information, ect.

%% TODO: Move this about, but after WebTorrent
\section{Tracker vs. Trackerless}
\label{sec:trackerless}
When running the BitTorrent protocol, we need a bootstrap mechanism, to get in
contact with peers within the network. Several bootstrap mechanisms are exist;
\begin{itemize}
\item Manual peer input
\item Tracker-based peer-lists
\item Trackerless peer-lists
\end{itemize}
The simplest model is that the user provide a number of peers, which are
already within the overlay network. These peers can then by used to further
bootstrap the network (see \citep{bittorrent:bep11}). These pre-known peers
can either be peers connected in the last session (and initially provided by
the software), or found by some means outside the network.

The tracker-based model utilizes a centralized component for acquiring the
initial list of peers to connect to. The flow is simply, a client asks the
tracker for a list of peers given a specific topic (aka. hash), after which the
client is able to contact the peers.

The trackerless model utilizes a \acs{DHT} network for acquiring the initial list of 
peers. The client joins the \acs{DHT} network (by knowing a peer(s) within the network),
and once inside the \acs{DHT} network, it can lookup topics (aka. hashes), and ask
the node responsible for the topic for the peer list.
\newline
- The known peer(s) required to join the \acs{DHT} network, are usually provided by
the \verb|.torrent| file for the download. A reference join node is presented
by BitTorrent, at \url{router.bittorrent.com}, but they recommend running your
own \citep{bittorrent:bep05}.



%This is where the author can answer the question of what use we can
%derive from all the works described in the previous chapter. Ideally,
%the summary of the related work will show that there is room
%unexplored for what the authors have in mind. If there are differences
%between the included works on key aspects in the approach to be taken,
%this is where this should be identified, and a decision reached.

%Having written the analysis, the author has all the tools needed to
%complete the next chapter.

\chapter{Design}
\label{cha:design}
% WebTorrent


% Audio Tag and limitations
Modern browsers already support native playback of audio files through the Audio \acs{HTML}5 tag, 
this capability comes with some significant limitations however; 
\newline

% Songs and Metadata


% LocalStorage vs LocalForage
\section{Fixing Browser Storage}
To have persistency in user sessions, we needed a way of storing data to the disk. There are several standard ways to store data in web browsers, but all of these have their own drawbacks.

Traditional browser cookies can store up to 4KB of data, 
not enough to save even one song. 
To store a song in cookies we would need to create thousands of cookies, 
and store small parts of a song in each one, 
this seemed like a difficult and inelegant solution.

Seeing the need for larger storage spaces, 
industry concerns have added LocalStorage to the \acs{HTML} 5 standard,
which allows websites to store arbitrarily large amounts of data,
but this again suffers significant drawbacks: 
implementation varies greatly between browsers, 
and can in some cases be wiped after reaching only 5 megabytes, 
hardly enough for our purposes.

Some browsers have also added support for database storage, via IndexedDB and MySQL, 
but this is not supported in every browser. 
For the sake of not excluding any browser that might work with WebTorrent, but not with IndexedDB,
it seemed like we needed a better solution.
\newline

To overcome these challenges, we use a library called localForage, 
which checks what the host browser supports,
and makes use of the best available storage method.
We found through experiments
that localForage has issues storing javascript objects to disk when these objects contain large binary blobs, 
these could consume as much as 4 gigabytes of space for an originally 5MB music file.
Seperating the generated metadata object and song binary before placing the in storage
seemed to solve this issue.
\newline

\section{Torrent Design}
% Granularity of torrents
A torrent is capable of containing one or many files of any format, 
it could contain single music files, whole albums or an artists entire discography.
Our system does not make use of existing torrents made by other applications,
so we are free to choose any granularity best suited for our purposes.
Most music services focus on individual songs, 
as users rarely wish to listen to multiple songs concurrently.
This seems to be the best choice for our project aswell, 
but the practicality of what to include in each of our torrents 
could also consider the performance of WebTorrent,
whether it can more efficiently make use of fewer torrents with more included files, 
or if this makes no difference.

%
%Many of the other natural sciences have labs with equipment that has
%to be configured correctly to experimentally test stated hypotheses.
%Such experiments must be planned and designed in advance to work
%properly and provide valid and trustworthy results.
%
%As computer scientists, we usually do not work in labs, and our
%experiments do not live in petri dishes. Still, we have hypotheses to
%test, and thus, experiments to plan. This planning phase is the
%design, where the authors describe the system intended to test the
%hypotheses posed in the introduction.
%
%A luxury of the design chapter is that the design may well go further
%than solely the confirmation or refutation of the hypotheses.  If you
%are building a system, this is where you show that you know how to
%design one, even if you will actually not be implementing all of it.
%If you had sufficient time and resources, this is how you would make
%your system.
%
%However, before we come to that, it is necessary to investigate
%whether the required hypotheses are valid. If they are not, the design
%must be reconsidered, and there is only one way to test them, namely
%through implementation, and subsequent evaluation.

\chapter{Implementation}
\label{cha:implementation}
\section{Overall Architecture}
% Code Architecture
Due to the size of the projects code-base and large number of libraries with different \acs{API}s, 
it was necessary to create different object classes with separated concerns,
and a consistent way to use them.

The code base is separated into a backend with code parts that are not visible to the user,
and a front end with each class having an associated \acs{GUI} element.
We also needed main classes to coordinate these responsibilities, handle overall flow,
and serve as a container for anything we did not have time to place in a separate class:
these classes are lib.ts in the backend,
and app.component.ts in the front-end to hold \acs{GUI} elements.

Many of the libraries used have slow, non-blocking IO operations
and make use of callbacks to execute code after the IO operations complete.

Most \acs{GUI} actions result in an event rather than an immediate code execution, 
these events are then caught by app.component
which performs the necessary functionality, through the other classes when relevant.

\begin{table}[h]
	\centering
	\begin{tabular}{l | l}
	    \verb|changing_song|    & User selected a different song to be played \\
		\verb|nextSong|         & User clicked the next song button \\
		\verb|prevSong|         & User clicked the previous song button \\
		\verb|song_ended|       & Song has ended and in line next should be played \\
		\verb|downloaded|       & Song has finished downloading \\
		\verb|ready-for-seed|   & Song is ready to be seeded \\
		\verb|add-song|         & Song added to playlist (from downloads) \\
		\verb|add-song|         & Song added to playlist (from localcontent) \\
		\verb|drop-down-select| & User clicked a search result \\
	\end{tabular}
	\caption{Events used in Streamy}
	\label{table:events}
\end{table}

When the system is started by the browser,
Angular loads the \acs{GUI} elements and then calls ngAfterViewInit in app.component.
ngAfterViewInit performs event bindings, and starts localcontent seeding.
Localcontent retrieves songs stored on the disk, and starts seeding them afterwards using a callback.
All other system functionality is triggered via these events, 
which are driven mostly by user actions and the music player.
\newline

% \acs{GUI} (Angular)
The \acs{GUI} needed data-bindings to show torrent information, allow changing songs
and implement a music player \acs{GUI}. 
Angular 2 lets us create \acs{GUI} elements as new \acs{HTML} tags, 
and bootstrap provides generic \acs{GUI} elements and stylesheets for angular, 
they were the easiest way to fill our needs for the \acs{GUI},
so that is what we have used.

Our \acs{GUI} elements consist of a music player interface, a play-list, downloading information 
and show information about local content and seeding.
\newline

\section{How we use Browserify}
% Browserify and why we use it
The project contains a large amount of NodeJS module libraries, 
many of which are designed for NodeJS and not for use in browsers.
We needed some way to use these libraries, and to insure that they are properly loaded.
In NodeJS, modules are loaded and made available by calling the require method 
and assigning the result to a field variable, which can then be used to access the libraries functions.
This capability is not supported natively in browsers, so a tool was needed to provide it.
The Browserify tool takes JavaScript code written for NodeJS, 
transforms it into regular JavaScript code, 
handles all the requires, 
and emits the new JavaScript code as a single easily distributed file called Bundle.js.
Browserify allows our projects end result to consist only of a \acs{HTML} page and one JavaScript file, 
so this seems like an ideal setup, and is what we make use of.
\newline

\section{Song handling}
% How we handle music and music metadata
The program encapsulates the highly inconsistent music files using a Song object,
which contains optional fields corresponding to commonly used meta-data, 
such as title, name of artist, genre and album. 
The Song object also contains information relevant for the BitTorrent system; 
it contains the magnetURI it was retrieved from or is being seeded to, 
it holds the binary file once it is fully available, 
or a reference to the music stream when it is still being downloaded.
This Song object presents all other sections of the project with a uniform way to access meta-data and audio, 
while retaining relational information about which albums or artists the song belongs to. In addition to this, 
we have also created objects to hold information about an album or an artist, 
so we can more easily search for related works in the \acs{DHT}.
\newline

\section{Torrenting}
% WebTorrent
WebTorrent provides seed and download methods using MagnetURI.
Individual downloads and seeds are controlled through a Torrent object provided by WebTorrent,
which contains all the relevant information about progress, 
speed, amount uploaded, number of peers, contained files and so on.
These torrent objects are provided by callbacks at events
and after each called WebTorrent function completes.
The encapsulating torrent.ts class gives access to seed and download methods of WebTorrent, 
but only for our Song Objects, which emphasis the projects intent of being a music streaming service 
rather than generic file sharing application, 
it also provides the WebTorrent torrent object through callbacks to other sections of our code.
\newline

\section{LocalContent \& Storage}
% Storage
The object-blob separation behavior and use of localforage,
needed to be consistent throughout the project, 
so we created a Storage class which uses localforage and handles the data separation implicitly when its storage methods are called.
We have also disallowed the direct use of localforage anywhere else:
all other sections of the project should save and get data through using the Storage class to ensure consistent behavior. 
This was done by removing any inclusion of the localforage library outside of the Storage class file.

As local content should also be visible in the \acs{GUI}, 
be seeded and be capable of being added to the playing,
we created another class called LocalContent,
which retrieves all songs from the disk at start-up,
begins to seed them, making them available for other users in the network to download,
and presents a \acs{GUI} element to the user.

\section{Tracker Implementation}
In our implementation, we've disregarded the manual peer input model, as it is
not user friendly. As such we're left with tracker-based and trackerless, as 
we've limited ourselves to the context of the browser, we're bound to using 
WebRTC, and by extension we need a centralized component to bootstrap these
connections. 

We are however, in theory, free to pick whether we want to use a tracker, or 
utilize a trackerless \acs{DHT} for bootstrapping peers, but only in theory as our
WebRTC-based BitTorrent implementation (as previously mentioned here) does
indeed not support trackerless, and as the provided WebSocket tracker for 
WebTorrent fulfills both the role of bootstrapping WebRTC connections, and the
role of the BitTorrent tracker.

As previously mentioned, had we been able to utilize trackerless, we've been
able to use the trackerless \acs{DHT} as our searching \acs{DHT}, however utilizing a
tracker provides other unique capabilities, namely; centralized control of the
swarm. 
\newline
- A content distributor might like to have this capability, to do data
processing on popular content, or otherwise track the swarm.

While a tracker-based solution utilizes a centralized components, it is not
limited to just using one. Our solution utilizes the 4 default trackers
provided by WebTorrent, but it's trivial to run your own tracker and add it to
the resources in the network.
\newline
- A content provider may want to track how often it's content is downloaded,
and could add the content to the network with just one tracker being their own.


%% TODO: Move this about
As previously stated we utilize a DHT overlay network for searching. While it's
currently faked, we do still add content as we would assuming we were utilizing
a non-faked implementation.

Whenever a song is uploaded, the following take place:
\begin{itemize}
\item We parse out the meta-data, and create a Song object from it.
\item We add the song to the DHT, with the key being the sha1 hash of it's title.
\item We check if the songs album is already within the DHT;
\begin{itemize}
\item If it is, we check if the artists are in correspondance, i.e. between
    this song, and the DHT content, if it is not, we update the information.
    Additionally, we add this song to the list of songs for the album.
\item If it is not, we add a new entry in the DHt, with the artists of this
    song, and add this song as the only song of the album.
\end{itemize}
\item We check if each of the artist(s) from the song, exists in the DHT.
\begin{itemize}
% TODO: Spell nessicary
\item If they do, we update their album information (if nessicary)
\item If they do not, we add them to the DHT.
\end{itemize}
\end{itemize}
In the above, we say that we add a X to Y, say a song to an album. The actual 
operation which takes place, is that we pull in the current JSON representation
of the album, then we add the song's title and it's hash (key into the DHT),
before pushing the modified JSON representation back into DHT.

In the faked implementation this is somewhat easy, while in a real DHT one
would have to implement a smarter updating scheme, i.e. to avoid race conditions.
\newline
% TODO: pattern name?
- A common scheme for databases is the transactional pattern, in which an update
either completely succeeds or fails without side-effects. We have not experimented
with implementing this sort of feature on top of a DHT network, but we envision
that a sub-protocol could be executed by the involved peers (i.e. the peers
responsible for the song, album and artists respectively).

Additionally the network would have to be protected against malicious updates.
We deem this outside the scope of our implementation, and as such we do not
provide a solution to this, but point towards assymmetric cryptography for a 
solutions.

A disclaimer has to be made; we (as previosly mentioned) currently fake the 
prefix hash trie implementation, as such we actually do some lookups which
bypasses the DHT interface. In order to utilize the same interface, we've
prefixed all sha1 hashes in the network with 'sha1:', and all queries that does
not have this prefix are supposed to be implemented on the prefix hash trie.


%Where the design chapter concerned itself with the overall plan, this
%is where the actual experiment in the form of an implementation is
%taking form.  It is not the purpose of the implementation to fully
%realise the design described in the previous chapter. It is the
%exclusive purpose of the implementation (a subset of the design) to
%either validate or refute the hypotheses put forth in the
%introduction. This, and nothing else. If it does less, you have posed
%questions you are not prepared to answer; if it does more, you should
%be coding less or asking more questions.
%
%If it illustrates core aspects, \eg, the inner working of a particular
%important algorithm or function, code segments are welcome in this
%chapter, as long as they are short, to the point, well-commented and
%-formatted.  It is also a good idea to provide the reader with a
%general overview of the structure of the code, as well as how
%communication between various parts take place.  The complete code (as
%well as your data) should be included separately with your report in
%the form of a zip-file or USB-stick.
%
%Overall, the implementation is the computer scientist's equivalent of
%lab equipment carefully arranged into a experimental setup, and just
%as the validity of an experimental investigation will be judged in
%part on the craftsmanship of the setup, so will the quality of your
%implementation. It is therefore important to clearly communicate how
%your system works, so that the reader may have confidence in your
%evaluation and conclusions.


\chapter{Evaluation}
\label{cha:evaluation}
%* Analysis results - Morten
%	- Capabilities compared to existing systems
%	- Is it useful/not useful?
%	- What does our system prove from research standpoint?
\section{results}

%not implemented
Because this project had a limited timeframe there is a couple of things we thing could be a good idea to have in the 
music streamer, but we did not have the time to implement.

\section{Seeder eviction}
When the storage capacity of a peer is full and the user want to get more songs, we need a strategy for what content we can throw away to get room for more.
An obvious choice is to run through the content we got from the bad health list and scrape it to see is all of those song are still at bad health. If there are many seeds on one of those songs we can throw the content away if we are not using it.
Another choice is to rn through all the songs we have downloaded and see if some of them is at good health but is seldomly played that could be thrown away.

\section{resource limitations}
If we run of storage while downloading a song we need to handle that problem.

%* What did we not implement? - Morten
%	- Seeder eviction
%	- Eviction strategies
%	- Resource limitations
%	- Bad Health supported properly?
%	- Missing features on GUI (why there are unused GUI source code)
%	- Fake DHT impl, what problems it causes
%	- Missing trackerless feature
%	- Could DHT and trackerless be implemented with more work?

%* Quantifiable Tests (Reproducible) (performance, robustness) - Morten
%	- CPU utilization when using/not using parts of project?
%	- LocalStorage performance degredation?
%	- Memory use?
%	- Comparison with spotify/others
%	- Reported traffic by webtorrent
%		Is it correct+complete?

%Having built the equivalent of a experimental setup, it is time to use
%the implementation to test the hypotheses.

\chapter{Conclusion}
\label{cha:conclusion}
% TODO: Write conclusion

\part{Demo}
\label{part:demo}
\chapter{Manual}
\label{cha:manual}
Here's a nice manual


\cleardoublepage

\chapter{Showcase}
Our project can be seen at: \url{https://youtu.be/X0lIZpOK1Qk}.
\newline
It is a description of where the music streamer can be accessed (for more
information see the manual (\ref{cha:manual}) for more information) and how to
use it.

% ********************************************************************
% Backmatter
%*******************************************************
\appendix
\cleardoublepage
\part{Appendix}
\chapter{Contributions}
\label{sec:appendix-contributions}
Throughout the project, we have developed in total 3 sub-projects;
\begin{itemize}
\item Angular2-interface
\item music-streamer-library
\item fake-dht
\end{itemize}

Below is a list of contributions to each of these projects, by group member:
\subsection{Angular2-interface}
% TODO: Content

\subsection{music-streamer-library}
% TODO: Content

\subsection{fake-dht}
% TODO: Content

\subsection{Other contributions}
In addition to these contributions, we've made several pull requests for the
libraries we depend on; namely to:
\begin{itemize}
\item music-metadata
\item Kad-WebRTC
\item localforage
\end{itemize}

Below is a list of contributions to each of these projects, by group member:
\subsubsection{music-metadata}
% TODO: Content

\subsubsection{Kad-WebRTC}
% TODO: Content

\subsubsection{localforage}
% TODO: Content


\chapter{Accepted project proposal}
\label{sec:appendix-project}
%% Project proposal
\centerline{
\includegraphics[page=1, scale=0.8]{gfx/project-proposal.pdf}
}

\centerline{
\includegraphics[page=2, scale=0.8]{gfx/project-proposal.pdf}
}


%********************************************************************
% Other Stuff in the Back
%*******************************************************
\cleardoublepage\include{FrontBackmatter/Bibliography}
\cleardoublepage%*******************************************************
% Declaration
%*******************************************************
\refstepcounter{dummy}
\pdfbookmark[0]{Declaration}{declaration}
\chapter*{Declaration}
\thispagestyle{empty}
Put your declaration here.
\bigskip
 
\noindent\textit{\myLocation, \myTime}

\smallskip

\begin{flushright}
    \begin{tabular}{m{5cm}}
        \\ \hline
        \centering\myNameOne \\
    \end{tabular}
\end{flushright}

\begin{flushright}
    \begin{tabular}{m{5cm}}
        \\ \hline
        \centering\myNameTwo \\
    \end{tabular}
\end{flushright}

\begin{flushright}
    \begin{tabular}{m{5cm}}
        \\ \hline
        \centering\myNameThree \\
    \end{tabular}
\end{flushright}

\end{document}
