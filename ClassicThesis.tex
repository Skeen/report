% **************************************************************************************************************
% A Classic Thesis Style
% An Homage to The Elements of Typographic Style
%
% Copyright (C) 2015 André Miede http://www.miede.de
%
% If you like the style then I would appreciate a postcard. My address 
% can be found in the file ClassicThesis.pdf. A collection of the 
% postcards I received so far is available online at 
% http://postcards.miede.de
%
% License:
% This program is free software; you can redistribute it and/or modify
% it under the terms of the GNU General Public License as published by
% the Free Software Foundation; either version 2 of the License, or
% (at your option) any later version.
%
% This program is distributed in the hope that it will be useful,
% but WITHOUT ANY WARRANTY; without even the implied warranty of
% MERCHANTABILITY or FITNESS FOR A PARTICULAR PURPOSE.  See the
% GNU General Public License for more details.
%
% You should have received a copy of the GNU General Public License
% along with this program; see the file COPYING.  If not, write to
% the Free Software Foundation, Inc., 59 Temple Place - Suite 330,
% Boston, MA 02111-1307, USA.
%
% **************************************************************************************************************
\RequirePackage{fix-cm} % fix some latex issues see: http://texdoc.net/texmf-dist/doc/latex/base/fixltx2e.pdf
\documentclass[ twoside,openright,titlepage,numbers=noenddot,headinclude,%1headlines,% letterpaper a4paper
                footinclude=true,cleardoublepage=empty,abstractoff, % <--- obsolete, remove (todo)
                BCOR=5mm,paper=a4,fontsize=11pt,%11pt,a4paper,%
                ngerman,american,%
                ]{scrreprt}

%********************************************************************
% Note: Make all your adjustments in here
%*******************************************************
\input{classicthesis-config}

%********************************************************************
% Bibliographies
%*******************************************************
\addbibresource{Bibliography.bib}
\addbibresource[label=ownpubs]{AMiede_Publications.bib}

%********************************************************************
% Hyphenation
%*******************************************************
%\hyphenation{put special hyphenation here}

% ********************************************************************
% GO!GO!GO! MOVE IT!
%*******************************************************
\begin{document}
\frenchspacing
\raggedbottom
\selectlanguage{american} % american ngerman
%\renewcommand*{\bibname}{new name}
%\setbibpreamble{}
\pagenumbering{roman}
\pagestyle{plain}
%********************************************************************
% Frontmatter
%*******************************************************
%*******************************************************
% Little Dirty Titlepage
%*******************************************************
\thispagestyle{empty}
%\pdfbookmark[1]{Titel}{title}
%*******************************************************
\vspace*{\fill}\noindent{%\rule{\linewidth}{1mm}\\[4ex]
{\Large \spacedallcaps \myTitle\\}\\[1ex]
{\large \spacedallcaps \mySubtitle}\\[2ex]
{\Large \spacedlowsmallcaps \myName, \myStudentId }\\[2ex]
\noindent\rule{\linewidth}{1mm}\\[4ex]
\noindent{\large \spacedlowsmallcaps \myDegree\\[1ex] 
\monthname\ \the\year  \\[1ex] Advisor: \myProf \\[23ex]}\\[\fill]}
\includegraphics[width=\linewidth]{gfx/logo}\clearpage


\include{FrontBackmatter/Titlepage}
\thispagestyle{empty}

\hfill

\vfill

\noindent\myNameOne, \myNameTwo, \myNameThree: \textit{\myTitle,} \myDegree,
\textcopyright\ \myTime

%\bigskip
%
%\noindent\spacedlowsmallcaps{Supervisors}: \\
%\myProf \\
%\myOtherProf \\ 
%\mySupervisor
%
%\medskip
%
%\noindent\spacedlowsmallcaps{Location}: \\
%\myLocation
%
%\medskip
%
%\noindent\spacedlowsmallcaps{Time Frame}: \\
%\myTime

\cleardoublepage%*******************************************************
% Dedication
%*******************************************************
\thispagestyle{empty}
%\phantomsection 
\refstepcounter{dummy}
\pdfbookmark[1]{Dedication}{Dedication}

\vspace*{3cm}

\begin{center}
    Sharing is good, and with digital technology, sharing is easy. \\ \medskip
    --- Richard Stallman
\end{center}

\medskip

\hfill

\vfill

\noindent\myNameOne, \myNameTwo, \myNameThree: \textit{\myTitle,} \myDegree,
\textcopyright\ \myTime



%\cleardoublepage\include{FrontBackmatter/Foreword}
\cleardoublepage%*******************************************************
% Abstract
%*******************************************************
%\renewcommand{\abstractname}{Abstract}
\pdfbookmark[1]{Abstract}{Abstract}
\begingroup
\let\clearpage\relax
\let\cleardoublepage\relax
\let\cleardoublepage\relax

\chapter*{Abstract}
Most mainstream multimedia content distribution networks and system, utilize
the traditional client-server architecture; as such scalability is by means of
larger and larger global datacenters, with ever increasing operation costs.
We've set forth the hypothesis that a fully decentralized multimedia content 
distribution network, with the properties of centralized ones is both feasible
and achiveable using modern software components.
Additionally we've limited ourselves to the somewhat limited execution
environment presented by the context of modern web-browsers, as we believe this
is the multimedia content distribution platform of the future.
We've developed a decentralized system for this platform, which utilizes a WebRTC
enabled BitTorrent implementation for low-level data transmission, and a 
(currently) faked prefix hash-trie for content indexing and searching.
We found that no proper prefix hash-trie (or even distributed hash table)
implementations which utilize WebRTC exists, and as such; we've been forced to
utilize a faked component for this (faked by centralization). As a result,
we're unable to utilize the system to prove our hypothesis, and we can only
provide our system as evidence towards the validity of it.

\vfill

\pdfbookmark[1]{Sammenfatning}{Sammenfatning}
\chapter*{Sammenfatning}
De fleste populære multimedie indholdsfordelingssystemer og netværk,
benytter den traditionelle klient-server arkitektur, hvormed systemet
udelukkende kan skaleres ved at bygge større og større globale data-centre
med evigt stigende udgifter til følge.
Vi har fremsat hypotesen at et fuldkomment decentraliseret multimedia 
indholdsfordelingssystem med egenskaberne fra et centraliseret system er både
realistisk og opnåeligt ved brug af moderne software komponenter.
Vi har begrænset os selv til at vores kode skal eksekveres i en moderne
browser, da vi vurdere dette som fremtidens multimedia indholdsfordelings
platform.
Vi har udviklet et decentraliseret system for denne platform. Systemet benytter
en BitTorrent implementering med et WebRTC transportlag for alle data
transmissioner, samt et (i øjeblikket forloren) prefix hash-trie for
indeksering og søgning efter indhold.
Da vi ikke har været i stand til at finde en ordenlig prefix hash-trie (eller
blot distributeret hash tabel) implementering, som benytter WebRTC har vi været
nødsaget til at benytte vores centralisered fake komponent. Dette betyder 
imidlertid at vi ikke kan benytte systemet direkte til at bekræfte vores 
hypotese, men må nøjes med at fremstille systemet som evidens for korrektheden
af denne.

\endgroup			

\vfill

\cleardoublepage\include{FrontBackmatter/Publications}
\cleardoublepage\include{FrontBackmatter/Acknowledgments}
\pagestyle{scrheadings}
\cleardoublepage\include{FrontBackmatter/Contents}
%********************************************************************
% Mainmatter
%*******************************************************
\cleardoublepage\pagenumbering{arabic}
%\setcounter{page}{90}
% use \cleardoublepage here to avoid problems with pdfbookmark
\cleardoublepage
% What should a thesis contain:

\ctparttext{The following reflects what I believe to be a good structure
  for a report or a thesis in experimental computer science. 
It contains a natural progression from the general to the specific, and
from the work of others to the work by the authors, each chapter
forming the foundation of the next.}
\part{The Proper Structure of a Thesis}
\label{part:prop-struct-thes}
\cleardoublepage

\chapter{Introduction}
\label{cha:introduction}

The purpose of the Introduction is make a short (2--6 pages) argument
that should cover
\begin{itemize}
\item What this thesis is about
\item Why it is interesting or important
\item What are the central hypotheses that will be investigated 
\item How will the work be done
\end{itemize}

This is the place where the reader (who will be a computer scientist,
but might not be a domain expert) should be convinced that not only is
the topic interesting and important, the authors have also identified
central questions/hypotheses pertaining the topic, and have a clear
plan and methodology to address it.

\section{What makes a good hypothesis?}
\label{sec:what-makes-good}

For the purposes of a report or thesis, it is wise to concentrate on
research questions and hypotheses that are quantifiable. \Eg, it is
better to state that ``method A is better than method B under
circumstances C'' or ``combining method A with architecture B improves
on standard approach C'' than ``we can build a system that do X''.
This is why it is always a good idea to include baselines in your
work, \ie, established methods or architectural choices that can used
for comparison. If you do not have baselines yourself, you should at
least be ready and able to compare your results with the published
results of others.

The hypotheses should also address central aspects of the work, so
that \emph{if} these hypotheses are met, the overall work gains in
credibility, or alternatively (and just as valid), if the hypothesis
\emph{cannot} be confirmed, it illustrates, why and how the
assumptions behind the work were flawed, and, hopefully, how they can
be improved.

\section{Writing a thesis for reading}
\label{sec:writ-thes-read}

The purpose of the thesis is to be read as a whole, and as such it
should be written, even if, in reality, it is authored over a period
of months.  The reader does not naturally understand the flow and
process of the work involved (this understanding belongs to the
authors, and upon the authors lies the sole responsibility of
communicating the work done), and must therefore be guided through the
work.  In order to accomplish this, the reader should at
all times have a ready answer in their mind to these questions:

\begin{itemize}
\item Why am I reading this?
\item What comes next?
\item How does this build upon what I just read?
\end{itemize}

So, why is something there? What is its purpose? How will it used
later? Vice versa, later in the text, refer back to things established
earlier (this also supports readers that do not necessarily read
linearly). While a text grow piecemeal, it is most often read as a
whole, and should appear as such, lest the reader loses interest.

To that end, it is a good idea to finish the introduction with a
description of how the hypotheses are to be investigated, and how this
is reflected in the structure of the thesis.

\chapter{Related Work}
\label{cha:related-work}

Whereas the purpose of the Introduction chapter was to entice and
convince the reader that work reported is interesting, that the author
is asking the right questions about it, and reading about it will be
worthwhile, the purpose of the Related Work chapter is to
demonstrate that the author possesses a fine overview and keen
understanding of the topic of the work.  Note that while the title of
the chapter is ``Related Work'', it might as well be called
``\emph{Relevant} Work'' in that you should only include work that are
useful or relevant to your purpose. 

Writing about others' work can be challenging---it is easy to succumb
to just writing condensed summaries, which is just as tedious to read
as they are to write. A better method is to gain an overview over the
field of inquiry, and then establish in the first section what aspects
or dimensions are crucial to systems or methodologies such as the ones
described. This demonstrates to the reader that the author has
understanding and judgement. Having done this, every paper or work can
then be described in those established terms. This makes for easier
and much more structured writing, and it also helps the reader
differentiate the systems and works reported on. If there are multiple
works that cover approximately the same area (\eg, using the same
technique), you may mention several, but only go into detail with the
most significant or representative one.

The chapter can then be concluded with a table summarising all the
work reported on using the aspects defined in the introduction of the
chapter.

A crucial element of this chapter is that it concerns the work of
others and \emph{only} that. While the selection of aspects or
dimensions described above invariantly will reflect your own focus,
that should be the extend of which your own work and plans influence
this chapter.  Your own judgement comes in the next chapter.

\section{Frameworks and Technologies}
\label{sec:fram-techn}

Related work need not be only published academic work. In many cases,
it is also relevant to describe crucial frameworks and technologies
that will be used or are relevant for the thesis.  This does not mean
that all employed technologies should be described in detail, but
frameworks and technologies that are unusual (for lack of a better
word) could be described here. \Eg, there is no need to describe an
ordinary network stack, but if the work involves GPU programming, a
description of the chosen architecture might well be relevant, as it
informs all the following chapters.



\chapter{Analysis}
\label{cha:analysis}

This is where the author can answer the question of what use we can
derive from all the works described in the previous chapter. Ideally,
the summary of the related work will show that there is room
unexplored for what the authors have in mind. If there are differences
between the included works on key aspects in the approach to be taken,
this is where this should be identified, and a decision reached.

Having written the analysis, the author has all the tools needed to
complete the next chapter.


\chapter{Design}
\label{cha:design}

Many of the other natural sciences have labs with equipment that has
to be configured correctly to experimentally test stated hypotheses.
Such experiments must be planned and designed in advance to work
properly and provide valid and trustworthy results.

As computer scientists, we usually do not work in labs, and our
experiments do not live in petri dishes. Still, we have hypotheses to
test, and thus, experiments to plan. This planning phase is the
design, where the authors describe the system intended to test the
hypotheses posed in the introduction.

A luxury of the design chapter is that the design may well go further
than solely the confirmation or refutation of the hypotheses.  If you
are building a system, this is where you show that you know how to
design one, even if you will actually not be implementing all of it.
If you had sufficient time and resources, this is how you would make
your system.

However, before we come to that, it is necessary to investigate
whether the required hypotheses are valid. If they are not, the design
must be reconsidered, and there is only one way to test them, namely
through implementation, and subsequent evaluation.


\chapter{Implementation}
\label{cha:implementation}

Where the design chapter concerned itself with the overall plan, this
is where the actual experiment in the form of an implementation is
taking form.  It is not the purpose of the implementation to fully
realise the design described in the previous chapter. It is the
exclusive purpose of the implementation (a subset of the design) to
either validate or refute the hypotheses put forth in the
introduction. This, and nothing else. If it does less, you have posed
questions you are not prepared to answer; if it does more, you should
be coding less or asking more questions.

If it illustrates core aspects, \eg, the inner working of a particular
important algorithm or function, code segments are welcome in this
chapter, as long as they are short, to the point, well-commented and
-formatted.  It is also a good idea to provide the reader with a
general overview of the structure of the code, as well as how
communication between various parts take place.  The complete code (as
well as your data) should be included separately with your report in
the form of a zip-file or USB-stick.

Overall, the implementation is the computer scientist's equivalent of
lab equipment carefully arranged into a experimental setup, and just
as the validity of an experimental investigation will be judged in
part on the craftsmanship of the setup, so will the quality of your
implementation. It is therefore important to clearly communicate how
your system works, so that the reader may have confidence in your
evaluation and conclusions.


\chapter{Evaluation}
\label{cha:evaluation}

Having built the equivalent of a experimental setup, it is time to use
the implementation to test the hypotheses.

This is usually broken down in stages and subquestions.

A structured approach to performing and reporting on experiments is
to follow this pattern for every single experiment:

\begin{enumerate}
\item What is the purpose of the experiment?
\item What is the expected outcome?
\item What are the parameters under which the experiment takes place?
\item What are the results?
\item How do the results align with the expected outcome? If they do
  not align, why is that so?
\end{enumerate}

Results should be presented summarised in tables and graphs.  Remember
to note the number of times experiments were repeated, as well as
averages, and standard deviations (in percent of the mean).  There is
much more to the proper evaluation of experimental data than can be
expounded upon here, but I turn the reader's attention to
\citep{Downey2011:TSPASFP2011}, which is freely available.


\chapter{Conclusion}
\label{cha:conclusion}

This, then is the grand summary of what you have accomplished.  You
may well imagine that many readers will read your Introduction, and
then skip to the Conclusion, and if, and only if, those two parts are
interesting, might be tempted to read the rest. A consequence is that
you should ensure that the reader will gain a good overall
understanding of what you have done by reading only the conclusion.
Thus, this is a place to summarise all that has gone before, before
finally concluding on the results of your experiments and the validity
of your hypotheses. It is also important to ensure that the
Introduction (which in all likelihood was written first) still aligns
closely with the conclusions reached.

If you so desire, this is also where you might add a section on Future
Work, where you point in the directions that should be followed to
complete the work you have already accomplished.



\part{Some Kind of Manual}
\include{Chapters/Chapter01}
\cleardoublepage

\part{The Showcase}
%BitTorrent
BitTorrent is a peer-to-peer protocol for file sharing designed by Bram Cohen in 2001.
The protocol is responsible for around 30 percent of the data uploaded to the internet.
The traditional way to ose the bittorrent protocol is to use a BitTorrent desktop client. That is a computer program that implements the BitTorrent protocol.
The .torrent files comes with a metadata file that includes a trackerlist. The bittorrent tracker is a server that contains information about what peers are interested in a given torrent. The tracker can connect a peer to other peers with the same torrent so the first peer can download and upload torrent data from and to those peers.


%tracker hash forbinde folk 
%bep
BitTorrent enhencement proposals is a place where users of BitTorrent can come with proposals to improve the protocol.
One of these proposals focuses on http seeding \citep{httpSeed}.
This is relevant in our work, because we focus on how we can make streaming work in the browser. The proposal is about changing the metadata file to include a httpseeds key. This key would refer to a list of web adresses where the torrent data can be downloaded from.

\section{Html5}
Html 5 is the latest markup language for writing web applications is was relaesed in 2014. Some of the new tags introduced is video and audio tags wich makes it possiple to play video and audio in the player build into html5.

Html5 also implements local storage. This makes it possible to store content locally in the browsers rather than use cookies, to store the data.
The storage limit it larger than when using cookies (at least 5mb) and the data is stored locally so it does not need to be sent to a server.

\section{browserify}
Browserify is a tool that makes it possible to use modules created to node.js in the browser. It also includes the 'require' functionallity in the browser code. That means code like this:
\begin{verbatim}
var EventEmitter = require('events').EventEmitter;
\end{verbatim}
will make sense in javascript now.

\chapter{Popular music streamers}
The are many music-streamers already reseased and we will let us get inspired by some of the most popular and give a walktrough of them below.

\section{spotify}
Spotify is the most popular online music streamer. The program was released in 2006 and was developed in Sweeden Stockholm by Daniel Ek and Martin Lorentzon. Now their main office is placed in Luxemborg and they have devisions in Stockholm and Göteborg. Their buisness model is that customers can listen to music in exchange for listening to commercials between tracks, or they can pay a monthly fee to be premium members.
The music is placed on servers controlled by spotify in contrast to the peer-to-peer music streamer we will develop.
\begin{figure}[p]
  \centering
  \includegraphics[width=0.9\textwidth]{gfx/Spotify_desktop.jpg}
  \label{spotify_desktop}
\end{figure}
As seen in \ref{spotify_desktop} the music player is placed in the buttom of the application, there is a menu at the left of the screen where your personal playlists can be accessed, and where it is possible to browse for new music and listen to a radio channel.
Spotify now have over 75 mission active users, both from Europe, America and Asia. Spotify both have clients for desktops and mobiles.

\section{Napster}
Napster was when it was released in 1999 a peer-to-peer file sharing service that focussed on audio files. Napster was co-founded by Shawn Fanning, John Fanning, and Sean Parker The Napster site was only operational untill 2001 because it was sued by Metallica for sharing music illigally. It was then brought down by court order. Metallica learned that their song "I dissapear" was avaiable on Napster before it was released witch also led to it be played on radios. On March 13, 2000 Metallica filed a lawsuit against Napster. In addition to the Napster weppage desktop client was develloped both to windows and later to MAC OS. 

%\addtocontents{toc}{\protect\clearpage} % <--- just debug stuff, ignore
\include{Chapters/Chapter03}



%\include{multiToC} % <--- just debug stuff, ignore for your documents
% ********************************************************************
% Backmatter
%*******************************************************
\appendix
%\renewcommand{\thechapter}{\alph{chapter}}
\cleardoublepage
\part{Appendix}
%********************************************************************
% Appendix
%*******************************************************
\chapter{Appendix}

%% Project proposal
\includepdf[pages=1, pagecommand={\section{Accepted project proposal}}]{gfx/project-proposal.pdf}

\includepdf[pages=2-,pagecommand={}]{gfx/project-proposal.pdf}

%********************************************************************
% Other Stuff in the Back
%*******************************************************
\cleardoublepage\include{FrontBackmatter/Bibliography}
\cleardoublepage%*******************************************************
% Declaration
%*******************************************************
\refstepcounter{dummy}
\pdfbookmark[0]{Declaration}{declaration}
\chapter*{Declaration}
\thispagestyle{empty}
Put your declaration here.
\bigskip
 
\noindent\textit{\myLocation, \myTime}

\smallskip

\begin{flushright}
    \begin{tabular}{m{5cm}}
        \\ \hline
        \centering\myNameOne \\
    \end{tabular}
\end{flushright}

\begin{flushright}
    \begin{tabular}{m{5cm}}
        \\ \hline
        \centering\myNameTwo \\
    \end{tabular}
\end{flushright}

\begin{flushright}
    \begin{tabular}{m{5cm}}
        \\ \hline
        \centering\myNameThree \\
    \end{tabular}
\end{flushright}

\cleardoublepage\include{FrontBackmatter/Colophon}
% ********************************************************************
% Game Over: Restore, Restart, or Quit?
%*******************************************************
\end{document}
% ********************************************************************
